% Generated by Sphinx.
\def\sphinxdocclass{report}
\documentclass[letterpaper,10pt,english]{sphinxmanual}
\usepackage[utf8]{inputenc}
\DeclareUnicodeCharacter{00A0}{\nobreakspace}
\usepackage[T1]{fontenc}
\usepackage{babel}
\usepackage{times}
\usepackage[Bjarne]{fncychap}
\usepackage{longtable}
\usepackage{sphinx}
\usepackage{multirow}


\title{pychoacoustics Documentation}
\date{July 04, 2013}
\release{('0.2.52',)}
\author{Samuele Carcagno}
\newcommand{\sphinxlogo}{}
\renewcommand{\releasename}{Release}
\makeindex

\makeatletter
\def\PYG@reset{\let\PYG@it=\relax \let\PYG@bf=\relax%
    \let\PYG@ul=\relax \let\PYG@tc=\relax%
    \let\PYG@bc=\relax \let\PYG@ff=\relax}
\def\PYG@tok#1{\csname PYG@tok@#1\endcsname}
\def\PYG@toks#1+{\ifx\relax#1\empty\else%
    \PYG@tok{#1}\expandafter\PYG@toks\fi}
\def\PYG@do#1{\PYG@bc{\PYG@tc{\PYG@ul{%
    \PYG@it{\PYG@bf{\PYG@ff{#1}}}}}}}
\def\PYG#1#2{\PYG@reset\PYG@toks#1+\relax+\PYG@do{#2}}

\expandafter\def\csname PYG@tok@gd\endcsname{\def\PYG@tc##1{\textcolor[rgb]{0.63,0.00,0.00}{##1}}}
\expandafter\def\csname PYG@tok@gu\endcsname{\let\PYG@bf=\textbf\def\PYG@tc##1{\textcolor[rgb]{0.50,0.00,0.50}{##1}}}
\expandafter\def\csname PYG@tok@gt\endcsname{\def\PYG@tc##1{\textcolor[rgb]{0.00,0.25,0.82}{##1}}}
\expandafter\def\csname PYG@tok@gs\endcsname{\let\PYG@bf=\textbf}
\expandafter\def\csname PYG@tok@gr\endcsname{\def\PYG@tc##1{\textcolor[rgb]{1.00,0.00,0.00}{##1}}}
\expandafter\def\csname PYG@tok@cm\endcsname{\let\PYG@it=\textit\def\PYG@tc##1{\textcolor[rgb]{0.25,0.50,0.56}{##1}}}
\expandafter\def\csname PYG@tok@vg\endcsname{\def\PYG@tc##1{\textcolor[rgb]{0.73,0.38,0.84}{##1}}}
\expandafter\def\csname PYG@tok@m\endcsname{\def\PYG@tc##1{\textcolor[rgb]{0.13,0.50,0.31}{##1}}}
\expandafter\def\csname PYG@tok@mh\endcsname{\def\PYG@tc##1{\textcolor[rgb]{0.13,0.50,0.31}{##1}}}
\expandafter\def\csname PYG@tok@cs\endcsname{\def\PYG@tc##1{\textcolor[rgb]{0.25,0.50,0.56}{##1}}\def\PYG@bc##1{\setlength{\fboxsep}{0pt}\colorbox[rgb]{1.00,0.94,0.94}{\strut ##1}}}
\expandafter\def\csname PYG@tok@ge\endcsname{\let\PYG@it=\textit}
\expandafter\def\csname PYG@tok@vc\endcsname{\def\PYG@tc##1{\textcolor[rgb]{0.73,0.38,0.84}{##1}}}
\expandafter\def\csname PYG@tok@il\endcsname{\def\PYG@tc##1{\textcolor[rgb]{0.13,0.50,0.31}{##1}}}
\expandafter\def\csname PYG@tok@go\endcsname{\def\PYG@tc##1{\textcolor[rgb]{0.19,0.19,0.19}{##1}}}
\expandafter\def\csname PYG@tok@cp\endcsname{\def\PYG@tc##1{\textcolor[rgb]{0.00,0.44,0.13}{##1}}}
\expandafter\def\csname PYG@tok@gi\endcsname{\def\PYG@tc##1{\textcolor[rgb]{0.00,0.63,0.00}{##1}}}
\expandafter\def\csname PYG@tok@gh\endcsname{\let\PYG@bf=\textbf\def\PYG@tc##1{\textcolor[rgb]{0.00,0.00,0.50}{##1}}}
\expandafter\def\csname PYG@tok@ni\endcsname{\let\PYG@bf=\textbf\def\PYG@tc##1{\textcolor[rgb]{0.84,0.33,0.22}{##1}}}
\expandafter\def\csname PYG@tok@nl\endcsname{\let\PYG@bf=\textbf\def\PYG@tc##1{\textcolor[rgb]{0.00,0.13,0.44}{##1}}}
\expandafter\def\csname PYG@tok@nn\endcsname{\let\PYG@bf=\textbf\def\PYG@tc##1{\textcolor[rgb]{0.05,0.52,0.71}{##1}}}
\expandafter\def\csname PYG@tok@no\endcsname{\def\PYG@tc##1{\textcolor[rgb]{0.38,0.68,0.84}{##1}}}
\expandafter\def\csname PYG@tok@na\endcsname{\def\PYG@tc##1{\textcolor[rgb]{0.25,0.44,0.63}{##1}}}
\expandafter\def\csname PYG@tok@nb\endcsname{\def\PYG@tc##1{\textcolor[rgb]{0.00,0.44,0.13}{##1}}}
\expandafter\def\csname PYG@tok@nc\endcsname{\let\PYG@bf=\textbf\def\PYG@tc##1{\textcolor[rgb]{0.05,0.52,0.71}{##1}}}
\expandafter\def\csname PYG@tok@nd\endcsname{\let\PYG@bf=\textbf\def\PYG@tc##1{\textcolor[rgb]{0.33,0.33,0.33}{##1}}}
\expandafter\def\csname PYG@tok@ne\endcsname{\def\PYG@tc##1{\textcolor[rgb]{0.00,0.44,0.13}{##1}}}
\expandafter\def\csname PYG@tok@nf\endcsname{\def\PYG@tc##1{\textcolor[rgb]{0.02,0.16,0.49}{##1}}}
\expandafter\def\csname PYG@tok@si\endcsname{\let\PYG@it=\textit\def\PYG@tc##1{\textcolor[rgb]{0.44,0.63,0.82}{##1}}}
\expandafter\def\csname PYG@tok@s2\endcsname{\def\PYG@tc##1{\textcolor[rgb]{0.25,0.44,0.63}{##1}}}
\expandafter\def\csname PYG@tok@vi\endcsname{\def\PYG@tc##1{\textcolor[rgb]{0.73,0.38,0.84}{##1}}}
\expandafter\def\csname PYG@tok@nt\endcsname{\let\PYG@bf=\textbf\def\PYG@tc##1{\textcolor[rgb]{0.02,0.16,0.45}{##1}}}
\expandafter\def\csname PYG@tok@nv\endcsname{\def\PYG@tc##1{\textcolor[rgb]{0.73,0.38,0.84}{##1}}}
\expandafter\def\csname PYG@tok@s1\endcsname{\def\PYG@tc##1{\textcolor[rgb]{0.25,0.44,0.63}{##1}}}
\expandafter\def\csname PYG@tok@gp\endcsname{\let\PYG@bf=\textbf\def\PYG@tc##1{\textcolor[rgb]{0.78,0.36,0.04}{##1}}}
\expandafter\def\csname PYG@tok@sh\endcsname{\def\PYG@tc##1{\textcolor[rgb]{0.25,0.44,0.63}{##1}}}
\expandafter\def\csname PYG@tok@ow\endcsname{\let\PYG@bf=\textbf\def\PYG@tc##1{\textcolor[rgb]{0.00,0.44,0.13}{##1}}}
\expandafter\def\csname PYG@tok@sx\endcsname{\def\PYG@tc##1{\textcolor[rgb]{0.78,0.36,0.04}{##1}}}
\expandafter\def\csname PYG@tok@bp\endcsname{\def\PYG@tc##1{\textcolor[rgb]{0.00,0.44,0.13}{##1}}}
\expandafter\def\csname PYG@tok@c1\endcsname{\let\PYG@it=\textit\def\PYG@tc##1{\textcolor[rgb]{0.25,0.50,0.56}{##1}}}
\expandafter\def\csname PYG@tok@kc\endcsname{\let\PYG@bf=\textbf\def\PYG@tc##1{\textcolor[rgb]{0.00,0.44,0.13}{##1}}}
\expandafter\def\csname PYG@tok@c\endcsname{\let\PYG@it=\textit\def\PYG@tc##1{\textcolor[rgb]{0.25,0.50,0.56}{##1}}}
\expandafter\def\csname PYG@tok@mf\endcsname{\def\PYG@tc##1{\textcolor[rgb]{0.13,0.50,0.31}{##1}}}
\expandafter\def\csname PYG@tok@err\endcsname{\def\PYG@bc##1{\setlength{\fboxsep}{0pt}\fcolorbox[rgb]{1.00,0.00,0.00}{1,1,1}{\strut ##1}}}
\expandafter\def\csname PYG@tok@kd\endcsname{\let\PYG@bf=\textbf\def\PYG@tc##1{\textcolor[rgb]{0.00,0.44,0.13}{##1}}}
\expandafter\def\csname PYG@tok@ss\endcsname{\def\PYG@tc##1{\textcolor[rgb]{0.32,0.47,0.09}{##1}}}
\expandafter\def\csname PYG@tok@sr\endcsname{\def\PYG@tc##1{\textcolor[rgb]{0.14,0.33,0.53}{##1}}}
\expandafter\def\csname PYG@tok@mo\endcsname{\def\PYG@tc##1{\textcolor[rgb]{0.13,0.50,0.31}{##1}}}
\expandafter\def\csname PYG@tok@mi\endcsname{\def\PYG@tc##1{\textcolor[rgb]{0.13,0.50,0.31}{##1}}}
\expandafter\def\csname PYG@tok@kn\endcsname{\let\PYG@bf=\textbf\def\PYG@tc##1{\textcolor[rgb]{0.00,0.44,0.13}{##1}}}
\expandafter\def\csname PYG@tok@o\endcsname{\def\PYG@tc##1{\textcolor[rgb]{0.40,0.40,0.40}{##1}}}
\expandafter\def\csname PYG@tok@kr\endcsname{\let\PYG@bf=\textbf\def\PYG@tc##1{\textcolor[rgb]{0.00,0.44,0.13}{##1}}}
\expandafter\def\csname PYG@tok@s\endcsname{\def\PYG@tc##1{\textcolor[rgb]{0.25,0.44,0.63}{##1}}}
\expandafter\def\csname PYG@tok@kp\endcsname{\def\PYG@tc##1{\textcolor[rgb]{0.00,0.44,0.13}{##1}}}
\expandafter\def\csname PYG@tok@w\endcsname{\def\PYG@tc##1{\textcolor[rgb]{0.73,0.73,0.73}{##1}}}
\expandafter\def\csname PYG@tok@kt\endcsname{\def\PYG@tc##1{\textcolor[rgb]{0.56,0.13,0.00}{##1}}}
\expandafter\def\csname PYG@tok@sc\endcsname{\def\PYG@tc##1{\textcolor[rgb]{0.25,0.44,0.63}{##1}}}
\expandafter\def\csname PYG@tok@sb\endcsname{\def\PYG@tc##1{\textcolor[rgb]{0.25,0.44,0.63}{##1}}}
\expandafter\def\csname PYG@tok@k\endcsname{\let\PYG@bf=\textbf\def\PYG@tc##1{\textcolor[rgb]{0.00,0.44,0.13}{##1}}}
\expandafter\def\csname PYG@tok@se\endcsname{\let\PYG@bf=\textbf\def\PYG@tc##1{\textcolor[rgb]{0.25,0.44,0.63}{##1}}}
\expandafter\def\csname PYG@tok@sd\endcsname{\let\PYG@it=\textit\def\PYG@tc##1{\textcolor[rgb]{0.25,0.44,0.63}{##1}}}

\def\PYGZbs{\char`\\}
\def\PYGZus{\char`\_}
\def\PYGZob{\char`\{}
\def\PYGZcb{\char`\}}
\def\PYGZca{\char`\^}
\def\PYGZam{\char`\&}
\def\PYGZlt{\char`\<}
\def\PYGZgt{\char`\>}
\def\PYGZsh{\char`\#}
\def\PYGZpc{\char`\%}
\def\PYGZdl{\char`\$}
\def\PYGZti{\char`\~}
% for compatibility with earlier versions
\def\PYGZat{@}
\def\PYGZlb{[}
\def\PYGZrb{]}
\makeatother

\begin{document}

\maketitle
\tableofcontents
\phantomsection\label{index::doc}


Contents:


\chapter{Indices and tables}
\label{index:indices-and-tables}\label{index:welcome-to-pychoacoustics-s-documentation}\begin{itemize}
\item {} 
\emph{genindex}

\item {} 
\emph{modindex}

\item {} 
\emph{search}

\end{itemize}


\chapter{\texttt{sndlib} -- Sound Synthesis Library}
\label{index:module-sndlib}\label{index:sndlib-sound-synthesis-library}\index{sndlib (module)}
A module for generating sounds in python.
\index{AMTone() (in module sndlib)}

\begin{fulllineitems}
\phantomsection\label{index:sndlib.AMTone}\pysiglinewithargsret{\code{sndlib.}\bfcode{AMTone}}{\emph{frequency}, \emph{AMFreq}, \emph{AMDepth}, \emph{phase}, \emph{level}, \emph{duration}, \emph{ramp}, \emph{channel}, \emph{fs}, \emph{maxLevel}}{}
Generate an amplitude modulated tone.
\begin{description}
\item[{frequency}] \leavevmode{[}float{]}
Carrier frequency in hertz.

\item[{AMFreq}] \leavevmode{[}float{]}
Amplitude modulation frequency in Hz.

\item[{AMDepth}] \leavevmode{[}float{]}
Amplitude modulation depth (a value of 1
corresponds to 100\% modulation).

\item[{phase}] \leavevmode{[}float{]}
Starting phase in radians.

\item[{level}] \leavevmode{[}float{]}
Tone level in dB SPL.

\item[{duration}] \leavevmode{[}float{]}
Tone duration (excluding ramps) in milliseconds.

\item[{ramp}] \leavevmode{[}float{]}
Duration of the onset and offset ramps in milliseconds.
The total duration of the sound will be duration+ramp*2.

\item[{channel}] \leavevmode{[}string (`Right', `Left' or `Both'){]}
Channel in which the tone will be generated.

\item[{fs}] \leavevmode{[}int{]}
Samplig frequency in Hz.

\item[{maxLevel}] \leavevmode{[}float{]}
Level in dB SPL output by the soundcard for a sinusoid of amplitude 1.

\end{description}

snd : 2-dimensional array of floats

\begin{Verbatim}[commandchars=\\\{\}]
\PYG{g+gp}{\PYGZgt{}\PYGZgt{}\PYGZgt{} }\PYG{n}{snd} \PYG{o}{=} \PYG{n}{AMTone}\PYG{p}{(}\PYG{n}{frequency}\PYG{o}{=}\PYG{l+m+mi}{1000}\PYG{p}{,} \PYG{n}{AMFreq}\PYG{o}{=}\PYG{l+m+mi}{20}\PYG{p}{,} \PYG{n}{AMDepth}\PYG{o}{=}\PYG{l+m+mi}{1}\PYG{p}{,} \PYG{n}{phase}\PYG{o}{=}\PYG{l+m+mi}{0}\PYG{p}{,} \PYG{n}{level}\PYG{o}{=}\PYG{l+m+mi}{65}\PYG{p}{,} \PYG{n}{duration}\PYG{o}{=}\PYG{l+m+mi}{180}\PYG{p}{,}
\PYG{g+gp}{... }    \PYG{n}{ramp}\PYG{o}{=}\PYG{l+m+mi}{10}\PYG{p}{,} \PYG{n}{channel}\PYG{o}{=}\PYG{l+s}{'}\PYG{l+s}{Both}\PYG{l+s}{'}\PYG{p}{,} \PYG{n}{fs}\PYG{o}{=}\PYG{l+m+mi}{48000}\PYG{p}{,} \PYG{n}{maxLevel}\PYG{o}{=}\PYG{l+m+mi}{100}\PYG{p}{)}
\end{Verbatim}

\end{fulllineitems}

\index{ERBDistance() (in module sndlib)}

\begin{fulllineitems}
\phantomsection\label{index:sndlib.ERBDistance}\pysiglinewithargsret{\code{sndlib.}\bfcode{ERBDistance}}{\emph{f1}, \emph{f2}}{}
\end{fulllineitems}

\index{FMTone() (in module sndlib)}

\begin{fulllineitems}
\phantomsection\label{index:sndlib.FMTone}\pysiglinewithargsret{\code{sndlib.}\bfcode{FMTone}}{\emph{fc}, \emph{fm}, \emph{mi}, \emph{phase}, \emph{level}, \emph{duration}, \emph{ramp}, \emph{channel}, \emph{fs}, \emph{maxLevel}}{}
Generate a frequency modulated tone.
\begin{description}
\item[{fc}] \leavevmode{[}float{]}
Carrier frequency in hertz. This is the frequency of the tone at fm zero crossing.

\item[{fm}] \leavevmode{[}float{]}
Modulation frequency in Hz.

\item[{mi}] \leavevmode{[}float{]}
Modulation index, also called beta and is equal to deltaF/fm, where
deltaF is the maximum deviation of the instantaneous frequency from
the carrier frequency.

\item[{phase}] \leavevmode{[}float{]}
Starting phase in radians.

\item[{level}] \leavevmode{[}float{]}
Tone level in dB SPL.

\item[{duration}] \leavevmode{[}float{]}
Tone duration (excluding ramps) in milliseconds.

\item[{ramp}] \leavevmode{[}float{]}
Duration of the onset and offset ramps in milliseconds.
The total duration of the sound will be duration+ramp*2.

\item[{channel}] \leavevmode{[}`Right', `Left' or `Both'{]}
Channel in which the tone will be generated.

\item[{fs}] \leavevmode{[}int{]}
Samplig frequency in Hz.

\item[{maxLevel}] \leavevmode{[}float{]}
Level in dB SPL output by the soundcard for a sinusoid of
amplitude 1.

\end{description}

snd : 2-dimensional array of floats

\begin{Verbatim}[commandchars=\\\{\}]
\PYG{g+gp}{\PYGZgt{}\PYGZgt{}\PYGZgt{} }\PYG{n}{snd} \PYG{o}{=} \PYG{n}{FMTone}\PYG{p}{(}\PYG{n}{fc}\PYG{o}{=}\PYG{l+m+mi}{1000}\PYG{p}{,} \PYG{n}{fm}\PYG{o}{=}\PYG{l+m+mi}{40}\PYG{p}{,} \PYG{n}{mi}\PYG{o}{=}\PYG{l+m+mi}{1}\PYG{p}{,} \PYG{n}{phase}\PYG{o}{=}\PYG{l+m+mi}{0}\PYG{p}{,} \PYG{n}{level}\PYG{o}{=}\PYG{l+m+mi}{55}\PYG{p}{,} \PYG{n}{duration}\PYG{o}{=}\PYG{l+m+mi}{180}\PYG{p}{,}
\PYG{g+gp}{... }    \PYG{n}{ramp}\PYG{o}{=}\PYG{l+m+mi}{10}\PYG{p}{,} \PYG{n}{channel}\PYG{o}{=}\PYG{l+s}{'}\PYG{l+s}{Both}\PYG{l+s}{'}\PYG{p}{,} \PYG{n}{fs}\PYG{o}{=}\PYG{l+m+mi}{48000}\PYG{p}{,} \PYG{n}{maxLevel}\PYG{o}{=}\PYG{l+m+mi}{100}\PYG{p}{)}
\end{Verbatim}

\end{fulllineitems}

\index{addSounds() (in module sndlib)}

\begin{fulllineitems}
\phantomsection\label{index:sndlib.addSounds}\pysiglinewithargsret{\code{sndlib.}\bfcode{addSounds}}{\emph{snd1}, \emph{snd2}, \emph{delay}, \emph{fs}}{}
Add or concatenate two sounds.
\begin{description}
\item[{snd1}] \leavevmode{[}array of floats{]}
First sound.

\item[{snd2}] \leavevmode{[}array of floats{]}
Second sound.

\item[{delay}] \leavevmode{[}float{]}
Delay in milliseconds between the onset of `snd1' and the onset of `snd2'

\item[{fs}] \leavevmode{[}float{]}
Sampling frequency in hertz of the two sounds.

\end{description}

snd : 2-dimensional array of floats

\begin{Verbatim}[commandchars=\\\{\}]
\PYG{g+gp}{\PYGZgt{}\PYGZgt{}\PYGZgt{} }\PYG{n}{snd1} \PYG{o}{=} \PYG{n}{pureTone}\PYG{p}{(}\PYG{n}{frequency}\PYG{o}{=}\PYG{l+m+mi}{440}\PYG{p}{,} \PYG{n}{phase}\PYG{o}{=}\PYG{l+m+mi}{0}\PYG{p}{,} \PYG{n}{level}\PYG{o}{=}\PYG{l+m+mi}{65}\PYG{p}{,} \PYG{n}{duration}\PYG{o}{=}\PYG{l+m+mi}{180}\PYG{p}{,}
\PYG{g+gp}{... }    \PYG{n}{ramp}\PYG{o}{=}\PYG{l+m+mi}{10}\PYG{p}{,} \PYG{n}{channel}\PYG{o}{=}\PYG{l+s}{'}\PYG{l+s}{Right}\PYG{l+s}{'}\PYG{p}{,} \PYG{n}{fs}\PYG{o}{=}\PYG{l+m+mi}{48000}\PYG{p}{,} \PYG{n}{maxLevel}\PYG{o}{=}\PYG{l+m+mi}{100}\PYG{p}{)}
\PYG{g+gp}{\PYGZgt{}\PYGZgt{}\PYGZgt{} }\PYG{n}{snd2} \PYG{o}{=} \PYG{n}{pureTone}\PYG{p}{(}\PYG{n}{frequency}\PYG{o}{=}\PYG{l+m+mi}{880}\PYG{p}{,} \PYG{n}{phase}\PYG{o}{=}\PYG{l+m+mi}{0}\PYG{p}{,} \PYG{n}{level}\PYG{o}{=}\PYG{l+m+mi}{65}\PYG{p}{,} \PYG{n}{duration}\PYG{o}{=}\PYG{l+m+mi}{180}\PYG{p}{,}
\PYG{g+gp}{... }    \PYG{n}{ramp}\PYG{o}{=}\PYG{l+m+mi}{10}\PYG{p}{,} \PYG{n}{channel}\PYG{o}{=}\PYG{l+s}{'}\PYG{l+s}{Right}\PYG{l+s}{'}\PYG{p}{,} \PYG{n}{fs}\PYG{o}{=}\PYG{l+m+mi}{48000}\PYG{p}{,} \PYG{n}{maxLevel}\PYG{o}{=}\PYG{l+m+mi}{100}\PYG{p}{)}
\PYG{g+gp}{\PYGZgt{}\PYGZgt{}\PYGZgt{} }\PYG{n}{snd} \PYG{o}{=} \PYG{n}{addSounds}\PYG{p}{(}\PYG{n}{snd1}\PYG{o}{=}\PYG{n}{snd1}\PYG{p}{,} \PYG{n}{snd2}\PYG{o}{=}\PYG{n}{snd2}\PYG{p}{,} \PYG{n}{delay}\PYG{o}{=}\PYG{l+m+mi}{100}\PYG{p}{,} \PYG{n}{fs}\PYG{o}{=}\PYG{l+m+mi}{48000}\PYG{p}{)}
\end{Verbatim}

\end{fulllineitems}

\index{binauralPureTone() (in module sndlib)}

\begin{fulllineitems}
\phantomsection\label{index:sndlib.binauralPureTone}\pysiglinewithargsret{\code{sndlib.}\bfcode{binauralPureTone}}{\emph{frequency}, \emph{phase}, \emph{level}, \emph{duration}, \emph{ramp}, \emph{channel}, \emph{itd}, \emph{itdRef}, \emph{ild}, \emph{ildRef}, \emph{fs}, \emph{maxLevel}}{}
Generate a pure tone with an optional interaural time or level difference.
\begin{description}
\item[{frequency}] \leavevmode{[}float{]}
Tone frequency in hertz.

\item[{phase}] \leavevmode{[}float{]}
Starting phase in radians.

\item[{level}] \leavevmode{[}float{]}
Tone level in dB SPL. If `ild' is different than zero, this will
be the level of the tone in the reference channel.

\item[{duration}] \leavevmode{[}float{]}
Tone duration (excluding ramps) in milliseconds.

\item[{ramp}] \leavevmode{[}float{]}
Duration of the onset and offset ramps in milliseconds.
The total duration of the sound will be duration+ramp*2.

\item[{channel}] \leavevmode{[}string (`Right', `Left' or `Both'){]}
Channel in which the tone will be generated.

\item[{itd}] \leavevmode{[}float{]}
Interaural time difference, in microseconds.

\item[{itdRef}] \leavevmode{[}`Right', `Left' or None{]}
The reference channel for the `itd'. The interaural time
difference will be applied to the other channel with
respect to the reference channel.

\item[{ild}] \leavevmode{[}float{]}
Interaural level difference in dB SPL.

\item[{ildRef}] \leavevmode{[}`Right', `Left' or None{]}
The reference channel for the `ild'.
The level of the other channel will be
icreased of attenuated by `ild' dB SPL
with respect to the reference channel.

\item[{fs}] \leavevmode{[}int{]}
Samplig frequency in Hz.

\item[{maxLevel}] \leavevmode{[}float{]}
Level in dB SPL output by the soundcard for a sinusoid of amplitude 1.

\end{description}
\begin{description}
\item[{snd}] \leavevmode{[}2-dimensional array of floats{]}
The array has dimensions (nSamples, 2).

\end{description}

\begin{Verbatim}[commandchars=\\\{\}]
\PYG{g+gp}{\PYGZgt{}\PYGZgt{}\PYGZgt{} }\PYG{n}{itdTone} \PYG{o}{=} \PYG{n}{binauralPureTone}\PYG{p}{(}\PYG{n}{frequency}\PYG{o}{=}\PYG{l+m+mi}{440}\PYG{p}{,} \PYG{n}{phase}\PYG{o}{=}\PYG{l+m+mi}{0}\PYG{p}{,} \PYG{n}{level}\PYG{o}{=}\PYG{l+m+mi}{65}\PYG{p}{,} \PYG{n}{duration}\PYG{o}{=}\PYG{l+m+mi}{180}\PYG{p}{,}
\PYG{g+gp}{... }    \PYG{n}{ramp}\PYG{o}{=}\PYG{l+m+mi}{10}\PYG{p}{,} \PYG{n}{channel}\PYG{o}{=}\PYG{l+s}{'}\PYG{l+s}{Both}\PYG{l+s}{'}\PYG{p}{,} \PYG{n}{itd}\PYG{o}{=}\PYG{l+m+mi}{480}\PYG{p}{,} \PYG{n}{itdRef}\PYG{o}{=}\PYG{l+s}{'}\PYG{l+s}{Right}\PYG{l+s}{'}\PYG{p}{,} \PYG{n}{ild}\PYG{o}{=}\PYG{l+m+mi}{0}\PYG{p}{,} \PYG{n}{ildRef}\PYG{o}{=}\PYG{n+nb+bp}{None}\PYG{p}{,}
\PYG{g+gp}{... }    \PYG{n}{fs}\PYG{o}{=}\PYG{l+m+mi}{48000}\PYG{p}{,} \PYG{n}{maxLevel}\PYG{o}{=}\PYG{l+m+mi}{100}\PYG{p}{)}
\PYG{g+gp}{\PYGZgt{}\PYGZgt{}\PYGZgt{} }\PYG{n}{ildTone} \PYG{o}{=} \PYG{n}{binauralPureTone}\PYG{p}{(}\PYG{n}{frequency}\PYG{o}{=}\PYG{l+m+mi}{440}\PYG{p}{,} \PYG{n}{phase}\PYG{o}{=}\PYG{l+m+mi}{0}\PYG{p}{,} \PYG{n}{level}\PYG{o}{=}\PYG{l+m+mi}{65}\PYG{p}{,} \PYG{n}{duration}\PYG{o}{=}\PYG{l+m+mi}{180}\PYG{p}{,}
\PYG{g+gp}{... }    \PYG{n}{ramp}\PYG{o}{=}\PYG{l+m+mi}{10}\PYG{p}{,} \PYG{n}{channel}\PYG{o}{=}\PYG{l+s}{'}\PYG{l+s}{Both}\PYG{l+s}{'}\PYG{p}{,} \PYG{n}{itd}\PYG{o}{=}\PYG{l+m+mi}{0}\PYG{p}{,} \PYG{n}{itdRef}\PYG{o}{=}\PYG{n+nb+bp}{None}\PYG{p}{,} \PYG{n}{ild}\PYG{o}{=}\PYG{o}{-}\PYG{l+m+mi}{20}\PYG{p}{,} \PYG{n}{ildRef}\PYG{o}{=}\PYG{l+s}{'}\PYG{l+s}{Right}\PYG{l+s}{'}\PYG{p}{,}
\PYG{g+gp}{... }    \PYG{n}{fs}\PYG{o}{=}\PYG{l+m+mi}{48000}\PYG{p}{,} \PYG{n}{maxLevel}\PYG{o}{=}\PYG{l+m+mi}{100}\PYG{p}{)}
\end{Verbatim}

\end{fulllineitems}

\index{broadbandNoise() (in module sndlib)}

\begin{fulllineitems}
\phantomsection\label{index:sndlib.broadbandNoise}\pysiglinewithargsret{\code{sndlib.}\bfcode{broadbandNoise}}{\emph{spectrumLevel}, \emph{duration}, \emph{ramp}, \emph{channel}, \emph{fs}, \emph{maxLevel}}{}
Synthetise a broadband noise.
\begin{description}
\item[{spectrumLevel}] \leavevmode{[}float{]}
Intensity spectrum level of the noise in dB SPL.

\item[{duration}] \leavevmode{[}float{]}
Noise duration (excluding ramps) in milliseconds.

\item[{ramp}] \leavevmode{[}float{]}
Duration of the onset and offset ramps in milliseconds.
The total duration of the sound will be duration+ramp*2.

\item[{channel}] \leavevmode{[}string (`Right', `Left' or `Both'){]}
Channel in which the noise will be generated.

\item[{fs}] \leavevmode{[}int{]}
Samplig frequency in Hz.

\item[{maxLevel}] \leavevmode{[}float{]}
Level in dB SPL output by the soundcard for a sinusoid of amplitude 1.

\end{description}
\begin{description}
\item[{snd}] \leavevmode{[}2-dimensional array of floats{]}
The array has dimensions (nSamples, 2).

\end{description}

\begin{Verbatim}[commandchars=\\\{\}]
\PYG{g+gp}{\PYGZgt{}\PYGZgt{}\PYGZgt{} }\PYG{n}{noise} \PYG{o}{=} \PYG{n}{broadbandNoise}\PYG{p}{(}\PYG{n}{spectrumLevel}\PYG{o}{=}\PYG{l+m+mi}{40}\PYG{p}{,} \PYG{n}{duration}\PYG{o}{=}\PYG{l+m+mi}{180}\PYG{p}{,} \PYG{n}{ramp}\PYG{o}{=}\PYG{l+m+mi}{10}\PYG{p}{,}
\PYG{g+gp}{... }    \PYG{n}{channel}\PYG{o}{=}\PYG{l+s}{'}\PYG{l+s}{Both}\PYG{l+s}{'}\PYG{p}{,} \PYG{n}{fs}\PYG{o}{=}\PYG{l+m+mi}{48000}\PYG{p}{,} \PYG{n}{maxLevel}\PYG{o}{=}\PYG{l+m+mi}{100}\PYG{p}{)}
\end{Verbatim}

\end{fulllineitems}

\index{camSinFMComplex() (in module sndlib)}

\begin{fulllineitems}
\phantomsection\label{index:sndlib.camSinFMComplex}\pysiglinewithargsret{\code{sndlib.}\bfcode{camSinFMComplex}}{\emph{F0}, \emph{lowHarm}, \emph{highHarm}, \emph{harmPhase}, \emph{fm}, \emph{deltaCams}, \emph{fmPhase}, \emph{level}, \emph{duration}, \emph{ramp}, \emph{channel}, \emph{fs}, \emph{maxLevel}}{}
Generate a tone frequency modulated with an exponential sinusoid.
\begin{description}
\item[{fc}] \leavevmode{[}float{]}
Carrier frequency in hertz.

\item[{fm}] \leavevmode{[}float{]}
Modulation frequency in Hz.

\item[{deltaCams}] \leavevmode{[}float{]}
Frequency excursion in cam units (ERBn number scale).

\item[{fmPhase}] \leavevmode{[}float{]}
Starting fmPhase in radians.

\item[{level}] \leavevmode{[}float{]}
Tone level in dB SPL.

\item[{duration}] \leavevmode{[}float{]}
Tone duration (excluding ramps) in milliseconds.

\item[{ramp}] \leavevmode{[}float{]}
Duration of the onset and offset ramps in milliseconds.
The total duration of the sound will be duration+ramp*2.

\item[{channel}] \leavevmode{[}`Right', `Left' or `Both'{]}
Channel in which the tone will be generated.

\item[{fs}] \leavevmode{[}int{]}
Samplig frequency in Hz.

\item[{maxLevel}] \leavevmode{[}float{]}
Level in dB SPL output by the soundcard for a sinusoid of
amplitude 1.

\end{description}

snd : 2-dimensional array of floats

\begin{Verbatim}[commandchars=\\\{\}]
\PYG{g+gp}{\PYGZgt{}\PYGZgt{}\PYGZgt{} }\PYG{n}{snd} \PYG{o}{=} \PYG{n}{expSinFMTone}\PYG{p}{(}\PYG{n}{fc}\PYG{o}{=}\PYG{l+m+mi}{1000}\PYG{p}{,} \PYG{n}{fm}\PYG{o}{=}\PYG{l+m+mi}{40}\PYG{p}{,} \PYG{n}{deltaCents}\PYG{o}{=}\PYG{l+m+mi}{1200}\PYG{p}{,} \PYG{n}{fmPhase}\PYG{o}{=}\PYG{l+m+mi}{0}\PYG{p}{,} \PYG{n}{level}\PYG{o}{=}\PYG{l+m+mi}{55}\PYG{p}{,} 
\PYG{g+gp}{... }    \PYG{n}{duration}\PYG{o}{=}\PYG{l+m+mi}{180}\PYG{p}{,} \PYG{n}{ramp}\PYG{o}{=}\PYG{l+m+mi}{10}\PYG{p}{,} \PYG{n}{channel}\PYG{o}{=}\PYG{l+s}{'}\PYG{l+s}{Both}\PYG{l+s}{'}\PYG{p}{,} \PYG{n}{fs}\PYG{o}{=}\PYG{l+m+mi}{48000}\PYG{p}{,} \PYG{n}{maxLevel}\PYG{o}{=}\PYG{l+m+mi}{100}\PYG{p}{)}
\end{Verbatim}

\end{fulllineitems}

\index{camSinFMTone() (in module sndlib)}

\begin{fulllineitems}
\phantomsection\label{index:sndlib.camSinFMTone}\pysiglinewithargsret{\code{sndlib.}\bfcode{camSinFMTone}}{\emph{fc}, \emph{fm}, \emph{deltaCams}, \emph{fmPhase}, \emph{startPhase}, \emph{level}, \emph{duration}, \emph{ramp}, \emph{channel}, \emph{fs}, \emph{maxLevel}}{}
Generate a tone frequency modulated with an exponential sinusoid.
\begin{description}
\item[{fc}] \leavevmode{[}float{]}
Carrier frequency in hertz.

\item[{fm}] \leavevmode{[}float{]}
Modulation frequency in Hz.

\item[{deltaCams}] \leavevmode{[}float{]}
Frequency excursion in cam units (ERBn number scale).

\item[{fmPhase}] \leavevmode{[}float{]}
Starting fmPhase in radians.

\item[{level}] \leavevmode{[}float{]}
Tone level in dB SPL.

\item[{duration}] \leavevmode{[}float{]}
Tone duration (excluding ramps) in milliseconds.

\item[{ramp}] \leavevmode{[}float{]}
Duration of the onset and offset ramps in milliseconds.
The total duration of the sound will be duration+ramp*2.

\item[{channel}] \leavevmode{[}`Right', `Left' or `Both'{]}
Channel in which the tone will be generated.

\item[{fs}] \leavevmode{[}int{]}
Samplig frequency in Hz.

\item[{maxLevel}] \leavevmode{[}float{]}
Level in dB SPL output by the soundcard for a sinusoid of
amplitude 1.

\end{description}

snd : 2-dimensional array of floats

\begin{Verbatim}[commandchars=\\\{\}]
\PYG{g+gp}{\PYGZgt{}\PYGZgt{}\PYGZgt{} }\PYG{n}{snd} \PYG{o}{=} \PYG{n}{expSinFMTone}\PYG{p}{(}\PYG{n}{fc}\PYG{o}{=}\PYG{l+m+mi}{1000}\PYG{p}{,} \PYG{n}{fm}\PYG{o}{=}\PYG{l+m+mi}{40}\PYG{p}{,} \PYG{n}{deltaCents}\PYG{o}{=}\PYG{l+m+mi}{1200}\PYG{p}{,} \PYG{n}{fmPhase}\PYG{o}{=}\PYG{l+m+mi}{0}\PYG{p}{,} \PYG{n}{level}\PYG{o}{=}\PYG{l+m+mi}{55}\PYG{p}{,} 
\PYG{g+gp}{... }    \PYG{n}{duration}\PYG{o}{=}\PYG{l+m+mi}{180}\PYG{p}{,} \PYG{n}{ramp}\PYG{o}{=}\PYG{l+m+mi}{10}\PYG{p}{,} \PYG{n}{channel}\PYG{o}{=}\PYG{l+s}{'}\PYG{l+s}{Both}\PYG{l+s}{'}\PYG{p}{,} \PYG{n}{fs}\PYG{o}{=}\PYG{l+m+mi}{48000}\PYG{p}{,} \PYG{n}{maxLevel}\PYG{o}{=}\PYG{l+m+mi}{100}\PYG{p}{)}
\end{Verbatim}

\end{fulllineitems}

\index{chirp() (in module sndlib)}

\begin{fulllineitems}
\phantomsection\label{index:sndlib.chirp}\pysiglinewithargsret{\code{sndlib.}\bfcode{chirp}}{\emph{freqStart}, \emph{ftype}, \emph{rate}, \emph{level}, \emph{duration}, \emph{phase}, \emph{ramp}, \emph{channel}, \emph{fs}, \emph{maxLevel}}{}
Synthetize a chirp, that is a tone with frequency changing linearly or
exponentially over time with a give rate.
\begin{description}
\item[{freqStart}] \leavevmode{[}float{]}
Starting frequency in hertz.

\item[{ftype}] \leavevmode{[}string{]}
If `linear', the frequency will change linearly on a Hz scale.
If `exponential', the frequency will change exponentially on a cents scale.

\item[{rate}] \leavevmode{[}float{]}
Rate of frequency change, Hz/s if ftype is `linear',
and cents/s if ftype is `exponential'.

\item[{level}] \leavevmode{[}float{]}
Level of the tone in dB SPL.

\item[{duration}] \leavevmode{[}float{]}
Tone duration (excluding ramps) in milliseconds.

\item[{ramp}] \leavevmode{[}float{]}
Duration of the onset and offset ramps in milliseconds.
The total duration of the sound will be duration+ramp*2.

\item[{channel}] \leavevmode{[}string (`Right', `Left' or `Both'){]}
Channel in which the tone will be generated.

\item[{fs}] \leavevmode{[}int{]}
Samplig frequency in Hz.

\item[{maxLevel}] \leavevmode{[}float{]}
Level in dB SPL output by the soundcard for a sinusoid of amplitude 1.

\end{description}
\begin{description}
\item[{snd}] \leavevmode{[}2-dimensional array of floats{]}
The array has dimensions (nSamples, 2).

\end{description}

\begin{Verbatim}[commandchars=\\\{\}]
\PYG{g+gp}{\PYGZgt{}\PYGZgt{}\PYGZgt{} }\PYG{n}{gl} \PYG{o}{=} \PYG{n}{chirp}\PYG{p}{(}\PYG{n}{freqStart}\PYG{o}{=}\PYG{l+m+mi}{440}\PYG{p}{,} \PYG{n}{ftype}\PYG{o}{=}\PYG{l+s}{'}\PYG{l+s}{linear}\PYG{l+s}{'}\PYG{p}{,} \PYG{n}{rate}\PYG{o}{=}\PYG{l+m+mi}{500}\PYG{p}{,} \PYG{n}{level}\PYG{o}{=}\PYG{l+m+mi}{55}\PYG{p}{,}
\PYG{g+go}{        duration=980, phase=0, ramp=10, channel='Both',}
\PYG{g+go}{        fs=48000, maxLevel=100)}
\end{Verbatim}

\end{fulllineitems}

\index{complexTone() (in module sndlib)}

\begin{fulllineitems}
\phantomsection\label{index:sndlib.complexTone}\pysiglinewithargsret{\code{sndlib.}\bfcode{complexTone}}{\emph{F0}, \emph{harmPhase}, \emph{lowHarm}, \emph{highHarm}, \emph{stretch}, \emph{level}, \emph{duration}, \emph{ramp}, \emph{channel}, \emph{fs}, \emph{maxLevel}}{}
Synthetise a complex tone.
\begin{description}
\item[{F0}] \leavevmode{[}float{]}
Tone fundamental frequency in hertz.

\item[{harmPhase}] \leavevmode{[}one of `Sine', `Cosine', `Alternating', `Random', `Schroeder'{]}
Phase relationship between the partials of the complex tone.

\item[{lowHarm}] \leavevmode{[}int{]}
Lowest harmonic component number.

\item[{highHarm}] \leavevmode{[}int{]}
Highest harmonic component number.

\item[{stretch}] \leavevmode{[}float{]}
Harmonic stretch in \%F0. Increase each harmonic frequency by a fixed value
that is equal to (F0*stretch)/100. If `stretch' is different than
zero, an inhanmonic complex tone will be generated.

\item[{level}] \leavevmode{[}float{]}
The level of each partial in dB SPL.

\item[{duration}] \leavevmode{[}float{]}
Tone duration (excluding ramps) in milliseconds.

\item[{ramp}] \leavevmode{[}float{]}
Duration of the onset and offset ramps in milliseconds.
The total duration of the sound will be duration+ramp*2.

\item[{channel}] \leavevmode{[}`Right', `Left', `Both', `Odd Right' or `Odd Left'{]}
Channel in which the tone will be generated. If `channel'
if `Odd Right', odd numbered harmonics will be presented
to the right channel and even number harmonics to the left
channel. The opposite is true if `channel' is `Odd Left'.

\item[{fs}] \leavevmode{[}int{]}
Samplig frequency in Hz.

\item[{maxLevel}] \leavevmode{[}float{]}
Level in dB SPL output by the soundcard for a sinusoid of amplitude 1.

\end{description}
\begin{description}
\item[{snd}] \leavevmode{[}2-dimensional array of floats{]}
The array has dimensions (nSamples, 2).

\end{description}

\begin{Verbatim}[commandchars=\\\{\}]
\PYG{g+gp}{\PYGZgt{}\PYGZgt{}\PYGZgt{} }\PYG{n}{ct} \PYG{o}{=} \PYG{n}{complexTone}\PYG{p}{(}\PYG{n}{F0}\PYG{o}{=}\PYG{l+m+mi}{440}\PYG{p}{,} \PYG{n}{harmPhase}\PYG{o}{=}\PYG{l+s}{'}\PYG{l+s}{Sine}\PYG{l+s}{'}\PYG{p}{,} \PYG{n}{lowHarm}\PYG{o}{=}\PYG{l+m+mi}{3}\PYG{p}{,} \PYG{n}{highHarm}\PYG{o}{=}\PYG{l+m+mi}{10}\PYG{p}{,}
\PYG{g+gp}{... }    \PYG{n}{stretch}\PYG{o}{=}\PYG{l+m+mi}{0}\PYG{p}{,} \PYG{n}{level}\PYG{o}{=}\PYG{l+m+mi}{55}\PYG{p}{,} \PYG{n}{duration}\PYG{o}{=}\PYG{l+m+mi}{180}\PYG{p}{,} \PYG{n}{ramp}\PYG{o}{=}\PYG{l+m+mi}{10}\PYG{p}{,} \PYG{n}{channel}\PYG{o}{=}\PYG{l+s}{'}\PYG{l+s}{Both}\PYG{l+s}{'}\PYG{p}{,}
\PYG{g+gp}{... }    \PYG{n}{fs}\PYG{o}{=}\PYG{l+m+mi}{48000}\PYG{p}{,} \PYG{n}{maxLevel}\PYG{o}{=}\PYG{l+m+mi}{100}\PYG{p}{)}
\end{Verbatim}

\end{fulllineitems}

\index{complexToneParallel() (in module sndlib)}

\begin{fulllineitems}
\phantomsection\label{index:sndlib.complexToneParallel}\pysiglinewithargsret{\code{sndlib.}\bfcode{complexToneParallel}}{\emph{F0}, \emph{harmPhase}, \emph{lowHarm}, \emph{highHarm}, \emph{stretch}, \emph{level}, \emph{duration}, \emph{ramp}, \emph{channel}, \emph{fs}, \emph{maxLevel}}{}
Synthetise a complex tone.

This function produces the same results of complexTone. The only difference
is that it uses the multiprocessing Python module to exploit multicore
processors and compute the partials in a parallel fashion. Notice that
there is a substantial overhead in setting up the parallel computations.
This means that for relatively short sounds (in the order of seconds),
this function will actually be \emph{slower} than complexTone.
\begin{description}
\item[{F0}] \leavevmode{[}float{]}
Tone fundamental frequency in hertz.

\item[{harmPhase}] \leavevmode{[}one of `Sine', `Cosine', `Alternating', `Random', `Schroeder'{]}
Phase relationship between the partials of the complex tone.

\item[{lowHarm}] \leavevmode{[}int{]}
Lowest harmonic component number.

\item[{highHarm}] \leavevmode{[}int{]}
Highest harmonic component number.

\item[{stretch}] \leavevmode{[}float{]}
Harmonic stretch in \%F0. Increase each harmonic frequency by a fixed value
that is equal to (F0*stretch)/100. If `stretch' is different than
zero, an inhanmonic complex tone will be generated.

\item[{level}] \leavevmode{[}float{]}
The level of each partial in dB SPL.

\item[{duration}] \leavevmode{[}float{]}
Tone duration (excluding ramps) in milliseconds.

\item[{ramp}] \leavevmode{[}float{]}
Duration of the onset and offset ramps in milliseconds.
The total duration of the sound will be duration+ramp*2.

\item[{channel}] \leavevmode{[}`Right', `Left', `Both', `Odd Right' or `Odd Left'{]}
Channel in which the tone will be generated. If `channel'
if `Odd Right', odd numbered harmonics will be presented
to the right channel and even number harmonics to the left
channel. The opposite is true if `channel' is `Odd Left'.

\item[{fs}] \leavevmode{[}int{]}
Samplig frequency in Hz.

\item[{maxLevel}] \leavevmode{[}float{]}
Level in dB SPL output by the soundcard for a sinusoid of amplitude 1.

\end{description}
\begin{description}
\item[{snd}] \leavevmode{[}2-dimensional array of floats{]}
The array has dimensions (nSamples, 2).

\end{description}

\begin{Verbatim}[commandchars=\\\{\}]
\PYG{g+gp}{\PYGZgt{}\PYGZgt{}\PYGZgt{} }\PYG{n}{ct} \PYG{o}{=} \PYG{n}{complexTone}\PYG{p}{(}\PYG{n}{F0}\PYG{o}{=}\PYG{l+m+mi}{440}\PYG{p}{,} \PYG{n}{harmPhase}\PYG{o}{=}\PYG{l+s}{'}\PYG{l+s}{Sine}\PYG{l+s}{'}\PYG{p}{,} \PYG{n}{lowHarm}\PYG{o}{=}\PYG{l+m+mi}{3}\PYG{p}{,} \PYG{n}{highHarm}\PYG{o}{=}\PYG{l+m+mi}{10}\PYG{p}{,}
\PYG{g+gp}{... }    \PYG{n}{stretch}\PYG{o}{=}\PYG{l+m+mi}{0}\PYG{p}{,} \PYG{n}{level}\PYG{o}{=}\PYG{l+m+mi}{55}\PYG{p}{,} \PYG{n}{duration}\PYG{o}{=}\PYG{l+m+mi}{180}\PYG{p}{,} \PYG{n}{ramp}\PYG{o}{=}\PYG{l+m+mi}{10}\PYG{p}{,} \PYG{n}{channel}\PYG{o}{=}\PYG{l+s}{'}\PYG{l+s}{Both}\PYG{l+s}{'}\PYG{p}{,}
\PYG{g+gp}{... }    \PYG{n}{fs}\PYG{o}{=}\PYG{l+m+mi}{48000}\PYG{p}{,} \PYG{n}{maxLevel}\PYG{o}{=}\PYG{l+m+mi}{100}\PYG{p}{)}
\end{Verbatim}

\end{fulllineitems}

\index{expAMNoise() (in module sndlib)}

\begin{fulllineitems}
\phantomsection\label{index:sndlib.expAMNoise}\pysiglinewithargsret{\code{sndlib.}\bfcode{expAMNoise}}{\emph{fc}, \emph{fm}, \emph{deltaCents}, \emph{fmPhase}, \emph{AMDepth}, \emph{spectrumLevel}, \emph{duration}, \emph{ramp}, \emph{channel}, \emph{fs}, \emph{maxLevel}}{}
Generate a sinusoidally amplitude-modulated noise with an exponentially
modulated AM frequency.
\begin{description}
\item[{fc}] \leavevmode{[}float{]}
Carrier AM frequency in hertz.

\item[{fm}] \leavevmode{[}float{]}
Modulation of the AM frequency in Hz.

\item[{deltaCents}] \leavevmode{[}float{]}\begin{description}
\item[{AM frequency excursion in cents. The instataneous AM frequency of the noise}] \leavevmode
will vary from fc**(-deltaCents/1200) to fc**(+deltaCents/1200).

\end{description}

\item[{fmPhase}] \leavevmode{[}float{]}
Starting phase of the AM modulation in radians.

\item[{AMDepth}] \leavevmode{[}float{]}
Amplitude modulation depth.

\item[{spectrumLevel}] \leavevmode{[}float{]}
Noise spectrum level in dB SPL.

\item[{duration}] \leavevmode{[}float{]}
Tone duration (excluding ramps) in milliseconds.

\item[{ramp}] \leavevmode{[}float{]}
Duration of the onset and offset ramps in milliseconds.
The total duration of the sound will be duration+ramp*2.

\item[{channel}] \leavevmode{[}`Right', `Left' or `Both'{]}
Channel in which the tone will be generated.

\item[{fs}] \leavevmode{[}int{]}
Samplig frequency in Hz.

\item[{maxLevel}] \leavevmode{[}float{]}
Level in dB SPL output by the soundcard for a sinusoid of
amplitude 1.

\end{description}

snd : 2-dimensional array of floats

\begin{Verbatim}[commandchars=\\\{\}]
\PYG{g+gp}{\PYGZgt{}\PYGZgt{}\PYGZgt{} }\PYG{n}{snd} \PYG{o}{=} \PYG{n}{expAMNoise}\PYG{p}{(}\PYG{n}{fc}\PYG{o}{=}\PYG{l+m+mi}{150}\PYG{p}{,} \PYG{n}{fm}\PYG{o}{=}\PYG{l+m+mf}{2.4}\PYG{p}{,} \PYG{n}{deltaCents}\PYG{o}{=}\PYG{l+m+mi}{1200}\PYG{p}{,} \PYG{n}{fmPhase}\PYG{o}{=}\PYG{l+m+mf}{3.14}\PYG{p}{,} \PYG{n}{AMDepth} \PYG{o}{=} \PYG{l+m+mi}{1}\PYG{p}{,}
\PYG{g+gp}{... }    \PYG{n}{spectrumLevel}\PYG{o}{=}\PYG{l+m+mi}{24}\PYG{p}{,} \PYG{n}{duration}\PYG{o}{=}\PYG{l+m+mi}{380}\PYG{p}{,} \PYG{n}{ramp}\PYG{o}{=}\PYG{l+m+mi}{10}\PYG{p}{,} \PYG{n}{channel}\PYG{o}{=}\PYG{l+s}{'}\PYG{l+s}{Both}\PYG{l+s}{'}\PYG{p}{,} \PYG{n}{fs}\PYG{o}{=}\PYG{l+m+mi}{48000}\PYG{p}{,} \PYG{n}{maxLevel}\PYG{o}{=}\PYG{l+m+mi}{100}\PYG{p}{)}
\end{Verbatim}

\end{fulllineitems}

\index{expSinFMComplex() (in module sndlib)}

\begin{fulllineitems}
\phantomsection\label{index:sndlib.expSinFMComplex}\pysiglinewithargsret{\code{sndlib.}\bfcode{expSinFMComplex}}{\emph{F0}, \emph{lowHarm}, \emph{highHarm}, \emph{harmPhase}, \emph{fm}, \emph{deltaCents}, \emph{fmPhase}, \emph{level}, \emph{duration}, \emph{ramp}, \emph{channel}, \emph{fs}, \emph{maxLevel}}{}
Generate a frequency-modulated complex tone with an exponential sinusoid.
\begin{description}
\item[{fc}] \leavevmode{[}float{]}
Carrier frequency in hertz.

\item[{fm}] \leavevmode{[}float{]}
Modulation frequency in Hz.

\item[{deltaCents}] \leavevmode{[}float{]}\begin{description}
\item[{Frequency excursion in cents. The instataneous frequency of the tone}] \leavevmode
will vary from fc**(-deltaCents/1200) to fc**(+deltaCents/1200).

\end{description}

\item[{fmPhase}] \leavevmode{[}float{]}
Starting fmPhase in radians.

\item[{level}] \leavevmode{[}float{]}
Tone level in dB SPL.

\item[{duration}] \leavevmode{[}float{]}
Tone duration (excluding ramps) in milliseconds.

\item[{ramp}] \leavevmode{[}float{]}
Duration of the onset and offset ramps in milliseconds.
The total duration of the sound will be duration+ramp*2.

\item[{channel}] \leavevmode{[}`Right', `Left' or `Both'{]}
Channel in which the tone will be generated.

\item[{fs}] \leavevmode{[}int{]}
Samplig frequency in Hz.

\item[{maxLevel}] \leavevmode{[}float{]}
Level in dB SPL output by the soundcard for a sinusoid of
amplitude 1.

\end{description}

snd : 2-dimensional array of floats

\begin{Verbatim}[commandchars=\\\{\}]
\PYG{g+gp}{\PYGZgt{}\PYGZgt{}\PYGZgt{} }\PYG{n}{snd} \PYG{o}{=} \PYG{n}{expSinFMTone}\PYG{p}{(}\PYG{n}{fc}\PYG{o}{=}\PYG{l+m+mi}{1000}\PYG{p}{,} \PYG{n}{fm}\PYG{o}{=}\PYG{l+m+mi}{40}\PYG{p}{,} \PYG{n}{deltaCents}\PYG{o}{=}\PYG{l+m+mi}{1200}\PYG{p}{,} \PYG{n}{fmPhase}\PYG{o}{=}\PYG{l+m+mi}{0}\PYG{p}{,} \PYG{n}{level}\PYG{o}{=}\PYG{l+m+mi}{55}\PYG{p}{,} 
\PYG{g+gp}{... }    \PYG{n}{duration}\PYG{o}{=}\PYG{l+m+mi}{180}\PYG{p}{,} \PYG{n}{ramp}\PYG{o}{=}\PYG{l+m+mi}{10}\PYG{p}{,} \PYG{n}{channel}\PYG{o}{=}\PYG{l+s}{'}\PYG{l+s}{Both}\PYG{l+s}{'}\PYG{p}{,} \PYG{n}{fs}\PYG{o}{=}\PYG{l+m+mi}{48000}\PYG{p}{,} \PYG{n}{maxLevel}\PYG{o}{=}\PYG{l+m+mi}{100}\PYG{p}{)}
\end{Verbatim}

\end{fulllineitems}

\index{expSinFMTone() (in module sndlib)}

\begin{fulllineitems}
\phantomsection\label{index:sndlib.expSinFMTone}\pysiglinewithargsret{\code{sndlib.}\bfcode{expSinFMTone}}{\emph{fc}, \emph{fm}, \emph{deltaCents}, \emph{fmPhase}, \emph{startPhase}, \emph{level}, \emph{duration}, \emph{ramp}, \emph{channel}, \emph{fs}, \emph{maxLevel}}{}
Generate a frequency-modulated tone with an exponential sinusoid.
\begin{description}
\item[{fc}] \leavevmode{[}float{]}
Carrier frequency in hertz.

\item[{fm}] \leavevmode{[}float{]}
Modulation frequency in Hz.

\item[{deltaCents}] \leavevmode{[}float{]}\begin{description}
\item[{Frequency excursion in cents. The instataneous frequency of the tone}] \leavevmode
will vary from fc**(-deltaCents/1200) to fc**(+deltaCents/1200).

\end{description}

\item[{fmPhase}] \leavevmode{[}float{]}
Starting fmPhase in radians.

\item[{level}] \leavevmode{[}float{]}
Tone level in dB SPL.

\item[{duration}] \leavevmode{[}float{]}
Tone duration (excluding ramps) in milliseconds.

\item[{ramp}] \leavevmode{[}float{]}
Duration of the onset and offset ramps in milliseconds.
The total duration of the sound will be duration+ramp*2.

\item[{channel}] \leavevmode{[}`Right', `Left' or `Both'{]}
Channel in which the tone will be generated.

\item[{fs}] \leavevmode{[}int{]}
Samplig frequency in Hz.

\item[{maxLevel}] \leavevmode{[}float{]}
Level in dB SPL output by the soundcard for a sinusoid of
amplitude 1.

\end{description}

snd : 2-dimensional array of floats

\begin{Verbatim}[commandchars=\\\{\}]
\PYG{g+gp}{\PYGZgt{}\PYGZgt{}\PYGZgt{} }\PYG{n}{snd} \PYG{o}{=} \PYG{n}{expSinFMTone}\PYG{p}{(}\PYG{n}{fc}\PYG{o}{=}\PYG{l+m+mi}{1000}\PYG{p}{,} \PYG{n}{fm}\PYG{o}{=}\PYG{l+m+mi}{40}\PYG{p}{,} \PYG{n}{deltaCents}\PYG{o}{=}\PYG{l+m+mi}{1200}\PYG{p}{,} \PYG{n}{fmPhase}\PYG{o}{=}\PYG{l+m+mi}{0}\PYG{p}{,} \PYG{n}{level}\PYG{o}{=}\PYG{l+m+mi}{55}\PYG{p}{,} 
\PYG{g+gp}{... }    \PYG{n}{duration}\PYG{o}{=}\PYG{l+m+mi}{180}\PYG{p}{,} \PYG{n}{ramp}\PYG{o}{=}\PYG{l+m+mi}{10}\PYG{p}{,} \PYG{n}{channel}\PYG{o}{=}\PYG{l+s}{'}\PYG{l+s}{Both}\PYG{l+s}{'}\PYG{p}{,} \PYG{n}{fs}\PYG{o}{=}\PYG{l+m+mi}{48000}\PYG{p}{,} \PYG{n}{maxLevel}\PYG{o}{=}\PYG{l+m+mi}{100}\PYG{p}{)}
\end{Verbatim}

\end{fulllineitems}

\index{fir2Filt() (in module sndlib)}

\begin{fulllineitems}
\phantomsection\label{index:sndlib.fir2Filt}\pysiglinewithargsret{\code{sndlib.}\bfcode{fir2Filt}}{\emph{f1}, \emph{f2}, \emph{f3}, \emph{f4}, \emph{snd}, \emph{fs}}{}
Filter signal with a fir2 filter.

This function designs and applies a fir2 filter to a sound.
The frequency response of the ideal filter will transition
from 0 to 1 between `f1' and `f2', and from 1 to zero
between `f3' and `f4'. The frequencies must be given in
increasing order.
\begin{description}
\item[{f1}] \leavevmode{[}float{]}
Frequency in hertz of the point at which the transition
for the low-frequency cutoff ends.

\item[{f2}] \leavevmode{[}float{]}
Frequency in hertz of the point at which the transition
for the low-frequency cutoff starts.

\item[{f3}] \leavevmode{[}float{]}
Frequency in hertz of the point at which the transition
for the high-frequency cutoff starts.

\item[{f4}] \leavevmode{[}float{]}
Frequency in hertz of the point at which the transition
for the high-frequency cutoff ends.

\item[{snd}] \leavevmode{[}array of floats{]}
The sound to be filtered.

\item[{fs}] \leavevmode{[}int{]}
Sampling frequency of `snd'.

\end{description}

snd : 2-dimensional array of floats

If `f1' and `f2' are zero the filter will be lowpass.
If `f3' and `f4' are equal to or greater than the nyquist
frequency (fs/2) the filter will be highpass.
In the other cases the filter will be bandpass.

The order of the filter (number of taps) is fixed at 256.
This function uses internally `scipy.signal.firwin2'.

\begin{Verbatim}[commandchars=\\\{\}]
\PYG{g+gp}{\PYGZgt{}\PYGZgt{}\PYGZgt{} }\PYG{n}{noise} \PYG{o}{=} \PYG{n}{broadbandNoise}\PYG{p}{(}\PYG{n}{spectrumLevel}\PYG{o}{=}\PYG{l+m+mi}{40}\PYG{p}{,} \PYG{n}{duration}\PYG{o}{=}\PYG{l+m+mi}{180}\PYG{p}{,} \PYG{n}{ramp}\PYG{o}{=}\PYG{l+m+mi}{10}\PYG{p}{,}
\PYG{g+gp}{... }    \PYG{n}{channel}\PYG{o}{=}\PYG{l+s}{'}\PYG{l+s}{Both}\PYG{l+s}{'}\PYG{p}{,} \PYG{n}{fs}\PYG{o}{=}\PYG{l+m+mi}{48000}\PYG{p}{,} \PYG{n}{maxLevel}\PYG{o}{=}\PYG{l+m+mi}{100}\PYG{p}{)}
\PYG{g+gp}{\PYGZgt{}\PYGZgt{}\PYGZgt{} }\PYG{n}{lpNoise} \PYG{o}{=} \PYG{n}{fir2Filt}\PYG{p}{(}\PYG{n}{f1}\PYG{o}{=}\PYG{l+m+mi}{0}\PYG{p}{,} \PYG{n}{f2}\PYG{o}{=}\PYG{l+m+mi}{0}\PYG{p}{,} \PYG{n}{f3}\PYG{o}{=}\PYG{l+m+mi}{1000}\PYG{p}{,} \PYG{n}{f4}\PYG{o}{=}\PYG{l+m+mi}{1200}\PYG{p}{,} 
\PYG{g+gp}{... }    \PYG{n}{snd}\PYG{o}{=}\PYG{n}{noise}\PYG{p}{,} \PYG{n}{fs}\PYG{o}{=}\PYG{l+m+mi}{48000}\PYG{p}{)} \PYG{c}{\PYGZsh{}lowpass filter}
\PYG{g+gp}{\PYGZgt{}\PYGZgt{}\PYGZgt{} }\PYG{n}{hpNoise} \PYG{o}{=} \PYG{n}{fir2Filt}\PYG{p}{(}\PYG{n}{f1}\PYG{o}{=}\PYG{l+m+mi}{0}\PYG{p}{,} \PYG{n}{f2}\PYG{o}{=}\PYG{l+m+mi}{0}\PYG{p}{,} \PYG{n}{f3}\PYG{o}{=}\PYG{l+m+mi}{24000}\PYG{p}{,} \PYG{n}{f4}\PYG{o}{=}\PYG{l+m+mi}{26000}\PYG{p}{,} 
\PYG{g+gp}{... }    \PYG{n}{snd}\PYG{o}{=}\PYG{n}{noise}\PYG{p}{,} \PYG{n}{fs}\PYG{o}{=}\PYG{l+m+mi}{48000}\PYG{p}{)} \PYG{c}{\PYGZsh{}highpass filter}
\PYG{g+gp}{\PYGZgt{}\PYGZgt{}\PYGZgt{} }\PYG{n}{bpNoise} \PYG{o}{=} \PYG{n}{fir2Filt}\PYG{p}{(}\PYG{n}{f1}\PYG{o}{=}\PYG{l+m+mi}{400}\PYG{p}{,} \PYG{n}{f2}\PYG{o}{=}\PYG{l+m+mi}{600}\PYG{p}{,} \PYG{n}{f3}\PYG{o}{=}\PYG{l+m+mi}{4000}\PYG{p}{,} \PYG{n}{f4}\PYG{o}{=}\PYG{l+m+mi}{4400}\PYG{p}{,} 
\PYG{g+gp}{... }    \PYG{n}{snd}\PYG{o}{=}\PYG{n}{noise}\PYG{p}{,} \PYG{n}{fs}\PYG{o}{=}\PYG{l+m+mi}{48000}\PYG{p}{)} \PYG{c}{\PYGZsh{}bandpass filter}
\end{Verbatim}

\end{fulllineitems}

\index{freqFromERBInterval() (in module sndlib)}

\begin{fulllineitems}
\phantomsection\label{index:sndlib.freqFromERBInterval}\pysiglinewithargsret{\code{sndlib.}\bfcode{freqFromERBInterval}}{\emph{f1}, \emph{deltaERB}}{}
\end{fulllineitems}

\index{gate() (in module sndlib)}

\begin{fulllineitems}
\phantomsection\label{index:sndlib.gate}\pysiglinewithargsret{\code{sndlib.}\bfcode{gate}}{\emph{ramps}, \emph{sig}, \emph{fs}}{}
Impose onset and offset ramps to a sound.
\begin{description}
\item[{ramps}] \leavevmode{[}float{]}
The duration of the ramps.

\item[{sig}] \leavevmode{[}array of floats    {]}
The signal on which the ramps should be imposed.

\item[{fs}] \leavevmode{[}int{]}
The sampling frequency os `sig'

\end{description}
\begin{description}
\item[{sig}] \leavevmode{[}array of floats{]}
The ramped signal.

\end{description}

\begin{Verbatim}[commandchars=\\\{\}]
\PYG{g+gp}{\PYGZgt{}\PYGZgt{}\PYGZgt{} }\PYG{n}{noise} \PYG{o}{=} \PYG{n}{broadbandNoise}\PYG{p}{(}\PYG{n}{spectrumLevel}\PYG{o}{=}\PYG{l+m+mi}{40}\PYG{p}{,} \PYG{n}{duration}\PYG{o}{=}\PYG{l+m+mi}{200}\PYG{p}{,} \PYG{n}{ramp}\PYG{o}{=}\PYG{l+m+mi}{0}\PYG{p}{,}
\PYG{g+gp}{... }    \PYG{n}{channel}\PYG{o}{=}\PYG{l+s}{'}\PYG{l+s}{Both}\PYG{l+s}{'}\PYG{p}{,} \PYG{n}{fs}\PYG{o}{=}\PYG{l+m+mi}{48000}\PYG{p}{,} \PYG{n}{maxLevel}\PYG{o}{=}\PYG{l+m+mi}{100}\PYG{p}{)}
\PYG{g+gp}{\PYGZgt{}\PYGZgt{}\PYGZgt{} }\PYG{n}{gate}\PYG{p}{(}\PYG{n}{ramps}\PYG{o}{=}\PYG{l+m+mi}{10}\PYG{p}{,} \PYG{n}{sig}\PYG{o}{=}\PYG{n}{noise}\PYG{p}{,} \PYG{n}{fs}\PYG{o}{=}\PYG{l+m+mi}{48000}\PYG{p}{)}
\end{Verbatim}

\end{fulllineitems}

\index{getRms() (in module sndlib)}

\begin{fulllineitems}
\phantomsection\label{index:sndlib.getRms}\pysiglinewithargsret{\code{sndlib.}\bfcode{getRms}}{\emph{sig}}{}
Compute the root mean square (RMS) value of the signal.
\begin{description}
\item[{sig}] \leavevmode{[}array of floats{]}
The signal for which the RMS needs to be computed.

\end{description}
\begin{description}
\item[{rms}] \leavevmode{[}float{]}
The RMS of `sig'.

\end{description}

\begin{Verbatim}[commandchars=\\\{\}]
\PYG{g+gp}{\PYGZgt{}\PYGZgt{}\PYGZgt{} }\PYG{n}{pt} \PYG{o}{=} \PYG{n}{pureTone}\PYG{p}{(}\PYG{n}{frequency}\PYG{o}{=}\PYG{l+m+mi}{440}\PYG{p}{,} \PYG{n}{phase}\PYG{o}{=}\PYG{l+m+mi}{0}\PYG{p}{,} \PYG{n}{level}\PYG{o}{=}\PYG{l+m+mi}{65}\PYG{p}{,} \PYG{n}{duration}\PYG{o}{=}\PYG{l+m+mi}{180}\PYG{p}{,}
\PYG{g+gp}{... }    \PYG{n}{ramp}\PYG{o}{=}\PYG{l+m+mi}{10}\PYG{p}{,} \PYG{n}{channel}\PYG{o}{=}\PYG{l+s}{'}\PYG{l+s}{Right}\PYG{l+s}{'}\PYG{p}{,} \PYG{n}{fs}\PYG{o}{=}\PYG{l+m+mi}{48000}\PYG{p}{,} \PYG{n}{maxLevel}\PYG{o}{=}\PYG{l+m+mi}{100}\PYG{p}{)}
\PYG{g+gp}{\PYGZgt{}\PYGZgt{}\PYGZgt{} }\PYG{n}{getRms}\PYG{p}{(}\PYG{n}{pt}\PYG{p}{)}
\end{Verbatim}

\end{fulllineitems}

\index{glide() (in module sndlib)}

\begin{fulllineitems}
\phantomsection\label{index:sndlib.glide}\pysiglinewithargsret{\code{sndlib.}\bfcode{glide}}{\emph{freqStart}, \emph{ftype}, \emph{excursion}, \emph{level}, \emph{duration}, \emph{phase}, \emph{ramp}, \emph{channel}, \emph{fs}, \emph{maxLevel}}{}
Synthetize a rising or falling tone glide with frequency changing
linearly or exponentially.
\begin{description}
\item[{freqStart}] \leavevmode{[}float{]}
Starting frequency in hertz.

\item[{ftype}] \leavevmode{[}string{]}
If `linear', the frequency will change linearly on a Hz scale.
If `exponential', the frequency will change exponentially on a cents scale.

\item[{excursion}] \leavevmode{[}float{]}
If ftype is `linear', excursion is the total frequency change in Hz.
The final frequency will be freqStart + excursion.
If ftype is `exponential', excursion is the total frequency change in cents.
The final frequency in Hz will be freqStart*2**(excusrion/1200).

\item[{level}] \leavevmode{[}float{]}
Level of the tone in dB SPL.

\item[{duration}] \leavevmode{[}float{]}
Tone duration (excluding ramps) in milliseconds.

\item[{ramp}] \leavevmode{[}float{]}
Duration of the onset and offset ramps in milliseconds.
The total duration of the sound will be duration+ramp*2.

\item[{channel}] \leavevmode{[}string (`Right', `Left' or `Both'){]}
Channel in which the tone will be generated.

\item[{fs}] \leavevmode{[}int{]}
Samplig frequency in Hz.

\item[{maxLevel}] \leavevmode{[}float{]}
Level in dB SPL output by the soundcard for a sinusoid of amplitude 1.

\end{description}
\begin{description}
\item[{snd}] \leavevmode{[}2-dimensional array of floats{]}
The array has dimensions (nSamples, 2).

\end{description}

\begin{Verbatim}[commandchars=\\\{\}]
\PYG{g+gp}{\PYGZgt{}\PYGZgt{}\PYGZgt{} }\PYG{n}{gl} \PYG{o}{=} \PYG{n}{glide}\PYG{p}{(}\PYG{n}{freqStart}\PYG{o}{=}\PYG{l+m+mi}{440}\PYG{p}{,} \PYG{n+nb}{type}\PYG{o}{=}\PYG{l+s}{'}\PYG{l+s}{exponential}\PYG{l+s}{'}\PYG{p}{,} \PYG{n}{excursion}\PYG{o}{=}\PYG{l+m+mi}{500}\PYG{p}{,}
\PYG{g+go}{        level=55, duration=180, phase=0, ramp=10, channel='Both',}
\PYG{g+go}{        fs=48000, maxLevel=100)}
\end{Verbatim}

\end{fulllineitems}

\index{harmComplFromNarrowbandNoise() (in module sndlib)}

\begin{fulllineitems}
\phantomsection\label{index:sndlib.harmComplFromNarrowbandNoise}\pysiglinewithargsret{\code{sndlib.}\bfcode{harmComplFromNarrowbandNoise}}{\emph{F0}, \emph{lowHarm}, \emph{highHarm}, \emph{level}, \emph{bandwidth}, \emph{duration}, \emph{ramp}, \emph{channel}, \emph{fs}, \emph{maxLevel}}{}
Generate an harmonic complex tone from narrow noise bands.
\begin{description}
\item[{F0}] \leavevmode{[}float{]}
Fundamental frequency of the complex.

\item[{lowHarm}] \leavevmode{[}int{]}
Lowest harmonic component number. The first component is \#1.

\item[{highHarm}] \leavevmode{[}int{]}
Highest harmonic component number.

\item[{level}] \leavevmode{[}float{]}
The spectrum level of the noise bands in dB SPL.

\item[{bandwidth}] \leavevmode{[}float{]}
The width of each noise band in hertz.

\item[{duration}] \leavevmode{[}float{]}
Tone duration (excluding ramps) in milliseconds.

\item[{ramp}] \leavevmode{[}float{]}
Duration of the onset and offset ramps in milliseconds.
The total duration of the sound will be duration+ramp*2.

\item[{channel}] \leavevmode{[}`Right', `Left', `Both', `Odd Right' or `Odd Left'{]}
Channel in which the tone will be generated. If `channel'
if `Odd Right', odd numbered harmonics will be presented
to the right channel and even number harmonics to the left
channel. The opposite is true if `channel' is `Odd Left'.

\item[{fs}] \leavevmode{[}int{]}
Samplig frequency in Hz.

\item[{maxLevel}] \leavevmode{[}float{]}
Level in dB SPL output by the soundcard for a sinusoid of amplitude 1.

\end{description}

snd : array of floats

\begin{Verbatim}[commandchars=\\\{\}]
\PYG{g+gp}{\PYGZgt{}\PYGZgt{}\PYGZgt{} }\PYG{n}{c1} \PYG{o}{=} \PYG{n}{harmComplFromNarrowbandNoise}\PYG{p}{(}\PYG{n}{F0}\PYG{o}{=}\PYG{l+m+mi}{440}\PYG{p}{,} \PYG{n}{lowHarm}\PYG{o}{=}\PYG{l+m+mi}{3}\PYG{p}{,} \PYG{n}{highHarm}\PYG{o}{=}\PYG{l+m+mi}{8}\PYG{p}{,}
\PYG{g+go}{     level=40, bandwidth=80, duration=180, ramp=10, channel='Both',}
\PYG{g+go}{     fs=48000, maxLevel=100)}
\end{Verbatim}

\end{fulllineitems}

\index{imposeLevelGlide() (in module sndlib)}

\begin{fulllineitems}
\phantomsection\label{index:sndlib.imposeLevelGlide}\pysiglinewithargsret{\code{sndlib.}\bfcode{imposeLevelGlide}}{\emph{sig}, \emph{deltaL}, \emph{startTime}, \emph{endTime}, \emph{channel}, \emph{fs}}{}
Impose a glide in level to a sound.

This function changes the level of a sound with a smooth transition (an amplitude
ramp) between `startTime' and `endTime'. If the signal input to the function
has a level L, the signal output by the function will have a level L
between time 0 and `startTime', and a level L+deltaL between endTime and
the end of the sound.
\begin{description}
\item[{sig}] \leavevmode{[}float{]}
Sound on which to impose the level change.

\item[{deltaL}] \leavevmode{[}float{]}
Magnitude of the level change in dB SPL.

\item[{startTime}] \leavevmode{[}float{]}
Start of the level transition in milliseconds.

\item[{endTime}] \leavevmode{[}float{]}
End of the level transition in milliseconds.

\item[{channel}] \leavevmode{[}string (`Right', `Left' or `Both'){]}
Channel to which apply the level transition.

\item[{fs}] \leavevmode{[}int{]}
Samplig frequency of the sound in Hz.

\end{description}

snd : array of floats

\begin{Verbatim}[commandchars=\\\{\}]
\PYG{g+gp}{\PYGZgt{}\PYGZgt{}\PYGZgt{} }\PYG{n}{pt} \PYG{o}{=} \PYG{n}{pureTone}\PYG{p}{(}\PYG{n}{frequency}\PYG{o}{=}\PYG{l+m+mi}{440}\PYG{p}{,} \PYG{n}{phase}\PYG{o}{=}\PYG{l+m+mi}{0}\PYG{p}{,} \PYG{n}{level}\PYG{o}{=}\PYG{l+m+mi}{65}\PYG{p}{,} \PYG{n}{duration}\PYG{o}{=}\PYG{l+m+mi}{180}\PYG{p}{,}
\PYG{g+gp}{... }    \PYG{n}{ramp}\PYG{o}{=}\PYG{l+m+mi}{10}\PYG{p}{,} \PYG{n}{channel}\PYG{o}{=}\PYG{l+s}{'}\PYG{l+s}{Right}\PYG{l+s}{'}\PYG{p}{,} \PYG{n}{fs}\PYG{o}{=}\PYG{l+m+mi}{48000}\PYG{p}{,} \PYG{n}{maxLevel}\PYG{o}{=}\PYG{l+m+mi}{100}\PYG{p}{)}
\PYG{g+gp}{\PYGZgt{}\PYGZgt{}\PYGZgt{} }\PYG{n}{pt2} \PYG{o}{=} \PYG{n}{imposeLevelGlide}\PYG{p}{(}\PYG{n}{sig}\PYG{o}{=}\PYG{n}{pt}\PYG{p}{,} \PYG{n}{deltaL}\PYG{o}{=}\PYG{l+m+mi}{10}\PYG{p}{,} \PYG{n}{startTime}\PYG{o}{=}\PYG{l+m+mi}{100}\PYG{p}{,}
\PYG{g+go}{        endTime=120, channel='Both', fs=48000)}
\end{Verbatim}

\end{fulllineitems}

\index{intNCyclesFreq() (in module sndlib)}

\begin{fulllineitems}
\phantomsection\label{index:sndlib.intNCyclesFreq}\pysiglinewithargsret{\code{sndlib.}\bfcode{intNCyclesFreq}}{\emph{freq}, \emph{duration}}{}
Compute the frequency closest to `freq' that has an integer number
of cycles for the given sound duration.
\begin{description}
\item[{frequency}] \leavevmode{[}float{]}
Frequency in hertz.

\item[{duration}] \leavevmode{[}float{]}
Duration of the sound, in milliseconds.

\end{description}

adjFreq : float

\begin{Verbatim}[commandchars=\\\{\}]
\PYG{g+gp}{\PYGZgt{}\PYGZgt{}\PYGZgt{} }\PYG{n}{intNCyclesFreq}\PYG{p}{(}\PYG{n}{freq}\PYG{o}{=}\PYG{l+m+mf}{2.1}\PYG{p}{,} \PYG{n}{duration}\PYG{o}{=}\PYG{l+m+mi}{1000}\PYG{p}{)}
\PYG{g+go}{2.0}
\PYG{g+gp}{\PYGZgt{}\PYGZgt{}\PYGZgt{} }\PYG{n}{intNCyclesFreq}\PYG{p}{(}\PYG{n}{freq}\PYG{o}{=}\PYG{l+m+mi}{2}\PYG{p}{,} \PYG{n}{duration}\PYG{o}{=}\PYG{l+m+mi}{1000}\PYG{p}{)}
\PYG{g+go}{2.0}
\end{Verbatim}

\end{fulllineitems}

\index{itdtoipd() (in module sndlib)}

\begin{fulllineitems}
\phantomsection\label{index:sndlib.itdtoipd}\pysiglinewithargsret{\code{sndlib.}\bfcode{itdtoipd}}{\emph{itd}, \emph{freq}}{}
Convert an interaural time difference to an equivalent interaural
phase difference for a given frequency.
\begin{description}
\item[{itd}] \leavevmode{[}float{]}
Interaural time difference in seconds.

\item[{freq}] \leavevmode{[}float{]}
Frequency in hertz.

\end{description}

ipd : float

\begin{Verbatim}[commandchars=\\\{\}]
\PYG{g+gp}{\PYGZgt{}\PYGZgt{}\PYGZgt{} }\PYG{n}{itd} \PYG{o}{=} \PYG{l+m+mi}{300} \PYG{c}{\PYGZsh{}microseconds}
\PYG{g+gp}{\PYGZgt{}\PYGZgt{}\PYGZgt{} }\PYG{n}{itd} \PYG{o}{=} \PYG{l+m+mi}{300}\PYG{o}{/}\PYG{l+m+mi}{1000000} \PYG{c}{\PYGZsh{}convert to seconds}
\PYG{g+gp}{\PYGZgt{}\PYGZgt{}\PYGZgt{} }\PYG{n}{itdtoipd}\PYG{p}{(}\PYG{n}{itd}\PYG{o}{=}\PYG{n}{itd}\PYG{p}{,} \PYG{n}{freq}\PYG{o}{=}\PYG{l+m+mi}{1000}\PYG{p}{)}
\end{Verbatim}

\end{fulllineitems}

\index{joinSndISI() (in module sndlib)}

\begin{fulllineitems}
\phantomsection\label{index:sndlib.joinSndISI}\pysiglinewithargsret{\code{sndlib.}\bfcode{joinSndISI}}{\emph{sndList}, \emph{ISIList}, \emph{fs}}{}
Join a list of sounds with given interstimulus intervals
\begin{description}
\item[{sndList}] \leavevmode{[}list of arrays{]}
The sounds to be joined.

\item[{ISIList}] \leavevmode{[}list of floats{]}
The interstimulus intervals between the sounds in milliseconds.
This list should have one element less than the sndList.

\item[{fs}] \leavevmode{[}int{]}
Sampling frequency of the sounds in Hz.

\end{description}

snd : array of floats

\begin{Verbatim}[commandchars=\\\{\}]
\PYG{g+gp}{\PYGZgt{}\PYGZgt{}\PYGZgt{} }\PYG{n}{pt1} \PYG{o}{=} \PYG{n}{pureTone}\PYG{p}{(}\PYG{n}{frequency}\PYG{o}{=}\PYG{l+m+mi}{440}\PYG{p}{,} \PYG{n}{phase}\PYG{o}{=}\PYG{l+m+mi}{0}\PYG{p}{,} \PYG{n}{level}\PYG{o}{=}\PYG{l+m+mi}{65}\PYG{p}{,} \PYG{n}{duration}\PYG{o}{=}\PYG{l+m+mi}{180}\PYG{p}{,}
\PYG{g+gp}{... }      \PYG{n}{ramp}\PYG{o}{=}\PYG{l+m+mi}{10}\PYG{p}{,} \PYG{n}{channel}\PYG{o}{=}\PYG{l+s}{'}\PYG{l+s}{Right}\PYG{l+s}{'}\PYG{p}{,} \PYG{n}{fs}\PYG{o}{=}\PYG{l+m+mi}{48000}\PYG{p}{,} \PYG{n}{maxLevel}\PYG{o}{=}\PYG{l+m+mi}{100}\PYG{p}{)}
\PYG{g+gp}{\PYGZgt{}\PYGZgt{}\PYGZgt{} }\PYG{n}{pt2} \PYG{o}{=} \PYG{n}{pureTone}\PYG{p}{(}\PYG{n}{frequency}\PYG{o}{=}\PYG{l+m+mi}{440}\PYG{p}{,} \PYG{n}{phase}\PYG{o}{=}\PYG{l+m+mi}{0}\PYG{p}{,} \PYG{n}{level}\PYG{o}{=}\PYG{l+m+mi}{65}\PYG{p}{,} \PYG{n}{duration}\PYG{o}{=}\PYG{l+m+mi}{180}\PYG{p}{,}
\PYG{g+gp}{... }      \PYG{n}{ramp}\PYG{o}{=}\PYG{l+m+mi}{10}\PYG{p}{,} \PYG{n}{channel}\PYG{o}{=}\PYG{l+s}{'}\PYG{l+s}{Right}\PYG{l+s}{'}\PYG{p}{,} \PYG{n}{fs}\PYG{o}{=}\PYG{l+m+mi}{48000}\PYG{p}{,} \PYG{n}{maxLevel}\PYG{o}{=}\PYG{l+m+mi}{100}\PYG{p}{)}
\PYG{g+gp}{\PYGZgt{}\PYGZgt{}\PYGZgt{} }\PYG{n}{tone\PYGZus{}seq} \PYG{o}{=} \PYG{n}{joinSndISI}\PYG{p}{(}\PYG{p}{[}\PYG{n}{pt1}\PYG{p}{,} \PYG{n}{pt2}\PYG{p}{]}\PYG{p}{,} \PYG{p}{[}\PYG{l+m+mi}{500}\PYG{p}{]}\PYG{p}{,} \PYG{l+m+mi}{48000}\PYG{p}{)}
\end{Verbatim}

\end{fulllineitems}

\index{makeAsynchChord() (in module sndlib)}

\begin{fulllineitems}
\phantomsection\label{index:sndlib.makeAsynchChord}\pysiglinewithargsret{\code{sndlib.}\bfcode{makeAsynchChord}}{\emph{freqs}, \emph{levels}, \emph{phases}, \emph{tonesDuration}, \emph{tonesRamps}, \emph{tonesChannel}, \emph{SOA}, \emph{fs}, \emph{maxLevel}}{}
Generate an asynchronous chord.

This function will add a set of pure tones with a given
stimulus onset asynchrony (SOA). The temporal order of the
successive tones is random.
\begin{description}
\item[{freqs}] \leavevmode{[}array or list of floats.{]}
Frequencies of the chord components in hertz.

\item[{levels}] \leavevmode{[}array or list of floats.{]}
Level of each chord component in dB SPL.

\item[{phases}] \leavevmode{[}array or list of floats.{]}
Starting phase of each chord component.

\item[{tonesDuration}] \leavevmode{[}float{]}
Duration of the tones composing the chord in milliseconds.
All tones have the same duration.

\item[{tonesRamps}] \leavevmode{[}float{]}
Duration of the onset and offset ramps in milliseconds.
The total duration of the tones will be tonesDuration+ramp*2.

\item[{tonesChannel}] \leavevmode{[}string (`Right', `Left' or `Both'){]}
Channel in which the tones will be generated.

\item[{SOA}] \leavevmode{[}float{]}
Onset asynchrony between the chord components.

\item[{fs}] \leavevmode{[}int{]}
Samplig frequency in Hz.

\item[{maxLevel}] \leavevmode{[}float{]}
Level in dB SPL output by the soundcard for a sinusoid of amplitude 1.

\end{description}

snd : 2-dimensional array of floats

\begin{Verbatim}[commandchars=\\\{\}]
\PYG{g+gp}{\PYGZgt{}\PYGZgt{}\PYGZgt{} }\PYG{n}{freqs} \PYG{o}{=} \PYG{p}{[}\PYG{l+m+mi}{250}\PYG{p}{,} \PYG{l+m+mi}{500}\PYG{p}{,} \PYG{l+m+mi}{1000}\PYG{p}{]}
\PYG{g+gp}{\PYGZgt{}\PYGZgt{}\PYGZgt{} }\PYG{n}{levels} \PYG{o}{=} \PYG{p}{[}\PYG{l+m+mi}{50}\PYG{p}{,} \PYG{l+m+mi}{50}\PYG{p}{,} \PYG{l+m+mi}{50}\PYG{p}{]}
\PYG{g+gp}{\PYGZgt{}\PYGZgt{}\PYGZgt{} }\PYG{n}{phases} \PYG{o}{=} \PYG{p}{[}\PYG{l+m+mi}{0}\PYG{p}{,} \PYG{l+m+mi}{0}\PYG{p}{,} \PYG{l+m+mi}{0}\PYG{p}{]}
\PYG{g+gp}{\PYGZgt{}\PYGZgt{}\PYGZgt{} }\PYG{n}{c1} \PYG{o}{=} \PYG{n}{makeAsynchChord}\PYG{p}{(}\PYG{n}{freqs}\PYG{o}{=}\PYG{n}{freqs}\PYG{p}{,} \PYG{n}{levels}\PYG{o}{=}\PYG{n}{levels}\PYG{p}{,} \PYG{n}{phases}\PYG{o}{=}\PYG{n}{phases}\PYG{p}{,}
\PYG{g+go}{        tonesDuration=180, tonesRamps=10, tonesChannel='Both',}
\PYG{g+go}{        SOA=60, fs=48000, maxLevel=100)}
\end{Verbatim}

\end{fulllineitems}

\index{makeHuggins() (in module sndlib)}

\begin{fulllineitems}
\phantomsection\label{index:sndlib.makeHuggins}\pysiglinewithargsret{\code{sndlib.}\bfcode{makeHuggins}}{\emph{F0}, \emph{lowHarm}, \emph{highHarm}, \emph{spectrumLevel}, \emph{bandwidth}, \emph{phaseRelationship}, \emph{noiseType}, \emph{duration}, \emph{ramp}, \emph{fs}, \emph{maxLevel}}{}
Synthetise a complex Huggings Pitch.
\begin{description}
\item[{F0}] \leavevmode{[}float{]}
The centre frequency of the F0 of the complex in hertz.

\item[{lowHarm}] \leavevmode{[}int{]}
Lowest harmonic component number.

\item[{highHarm}] \leavevmode{[}int{]}
Highest harmonic component number.

\item[{spectrumLevel}] \leavevmode{[}float{]}
The spectrum level of the noise from which
the complex is derived in dB SPL.

\item[{bandwidth}] \leavevmode{[}float{]}
Bandwidth of the frequency regions in which the
phase transitions occurr.

\item[{phaseRelationship}] \leavevmode{[}string (`NoSpi' or `NpiSo'){]}
If NoSpi, the phase of the regions within each frequency band will
be shifted. If NpiSo, the phase of the regions between each
frequency band will be shifted.

\item[{noiseType}] \leavevmode{[}string (`White' or `Pink'){]}
The type of noise used to derive the Huggins Pitch.

\item[{duration}] \leavevmode{[}float{]}
Complex duration (excluding ramps) in milliseconds.

\item[{ramp}] \leavevmode{[}float{]}
Duration of the onset and offset ramps in milliseconds.
The total duration of the sound will be duration+ramp*2.

\item[{fs}] \leavevmode{[}int{]}
Samplig frequency in Hz.

\item[{maxLevel}] \leavevmode{[}float{]}
Level in dB SPL output by the soundcard for a sinusoid of amplitude 1.

\end{description}
\begin{description}
\item[{snd}] \leavevmode{[}2-dimensional array of floats{]}
The array has dimensions (nSamples, 2).

\end{description}

\begin{Verbatim}[commandchars=\\\{\}]
\PYG{g+gp}{\PYGZgt{}\PYGZgt{}\PYGZgt{} }\PYG{n}{hp} \PYG{o}{=} \PYG{n}{makeHuggins}\PYG{p}{(}\PYG{n}{F0}\PYG{o}{=}\PYG{l+m+mi}{200}\PYG{p}{,} \PYG{n}{lowHarm}\PYG{o}{=}\PYG{l+m+mi}{1}\PYG{p}{,} \PYG{n}{highHarm}\PYG{o}{=}\PYG{l+m+mi}{5}\PYG{p}{,} \PYG{n}{spectrumLevel}\PYG{o}{=}\PYG{l+m+mi}{40}\PYG{p}{,}
\PYG{g+go}{        bandwidth=65, phaseRelationship='NoSpi', noiseType='White',}
\PYG{g+go}{        duration=280, ramp=10, fs=48000, maxLevel=100)}
\end{Verbatim}

\end{fulllineitems}

\index{makePink() (in module sndlib)}

\begin{fulllineitems}
\phantomsection\label{index:sndlib.makePink}\pysiglinewithargsret{\code{sndlib.}\bfcode{makePink}}{\emph{sig}, \emph{fs}}{}
Convert a white noise into a pink noise.

The spectrum level of the pink noise at 1000 Hz will be equal to
the spectrum level of the white noise input to the function.
\begin{description}
\item[{sig}] \leavevmode{[}array of floats{]}
The white noise to be turned into a pink noise.

\item[{fs}] \leavevmode{[}int{]}
Sampling frequency of the sound.

\end{description}
\begin{description}
\item[{snd}] \leavevmode{[}2-dimensional array of floats{]}
The array has dimensions (nSamples, 2).

\end{description}

\begin{Verbatim}[commandchars=\\\{\}]
\PYG{g+gp}{\PYGZgt{}\PYGZgt{}\PYGZgt{} }\PYG{n}{noise} \PYG{o}{=} \PYG{n}{broadbandNoise}\PYG{p}{(}\PYG{n}{spectrumLevel}\PYG{o}{=}\PYG{l+m+mi}{40}\PYG{p}{,} \PYG{n}{duration}\PYG{o}{=}\PYG{l+m+mi}{180}\PYG{p}{,} \PYG{n}{ramp}\PYG{o}{=}\PYG{l+m+mi}{10}\PYG{p}{,}
\PYG{g+gp}{... }    \PYG{n}{channel}\PYG{o}{=}\PYG{l+s}{'}\PYG{l+s}{Both}\PYG{l+s}{'}\PYG{p}{,} \PYG{n}{fs}\PYG{o}{=}\PYG{l+m+mi}{48000}\PYG{p}{,} \PYG{n}{maxLevel}\PYG{o}{=}\PYG{l+m+mi}{100}\PYG{p}{)}
\PYG{g+gp}{\PYGZgt{}\PYGZgt{}\PYGZgt{} }\PYG{n}{noise} \PYG{o}{=} \PYG{n}{makePink}\PYG{p}{(}\PYG{n}{sig}\PYG{o}{=}\PYG{n}{noise}\PYG{p}{,} \PYG{n}{fs}\PYG{o}{=}\PYG{l+m+mi}{48000}\PYG{p}{)}
\end{Verbatim}

\end{fulllineitems}

\index{makePinkRef() (in module sndlib)}

\begin{fulllineitems}
\phantomsection\label{index:sndlib.makePinkRef}\pysiglinewithargsret{\code{sndlib.}\bfcode{makePinkRef}}{\emph{sig}, \emph{fs}, \emph{refHz}}{}
Convert a white noise into a pink noise.

The spectrum level of the pink noise at the frequency `refHz'
will be equal to the spectrum level of the white noise input
to the function.
\begin{description}
\item[{sig}] \leavevmode{[}array of floats{]}
The white noise to be turned into a pink noise.

\item[{fs}] \leavevmode{[}int{]}
Sampling frequency of the sound.

\item[{refHz}] \leavevmode{[}int{]}
Reference frequency in Hz. The amplitude of the other
frequencies will be scaled with respect to the amplitude
of this frequency.

\end{description}
\begin{description}
\item[{snd}] \leavevmode{[}2-dimensional array of floats{]}
The array has dimensions (nSamples, 2).

\end{description}

\begin{Verbatim}[commandchars=\\\{\}]
\PYG{g+gp}{\PYGZgt{}\PYGZgt{}\PYGZgt{} }\PYG{n}{noise} \PYG{o}{=} \PYG{n}{broadbandNoise}\PYG{p}{(}\PYG{n}{spectrumLevel}\PYG{o}{=}\PYG{l+m+mi}{40}\PYG{p}{,} \PYG{n}{duration}\PYG{o}{=}\PYG{l+m+mi}{180}\PYG{p}{,} \PYG{n}{ramp}\PYG{o}{=}\PYG{l+m+mi}{10}\PYG{p}{,}
\PYG{g+gp}{... }    \PYG{n}{channel}\PYG{o}{=}\PYG{l+s}{'}\PYG{l+s}{Both}\PYG{l+s}{'}\PYG{p}{,} \PYG{n}{fs}\PYG{o}{=}\PYG{l+m+mi}{48000}\PYG{p}{,} \PYG{n}{maxLevel}\PYG{o}{=}\PYG{l+m+mi}{100}\PYG{p}{)}
\PYG{g+gp}{\PYGZgt{}\PYGZgt{}\PYGZgt{} }\PYG{n}{noise} \PYG{o}{=} \PYG{n}{makePink}\PYG{p}{(}\PYG{n}{sig}\PYG{o}{=}\PYG{n}{noise}\PYG{p}{,} \PYG{n}{fs}\PYG{o}{=}\PYG{l+m+mi}{48000}\PYG{p}{,} \PYG{n}{refHz}\PYG{o}{=}\PYG{l+m+mi}{1000}\PYG{p}{)}
\end{Verbatim}

\end{fulllineitems}

\index{makeSilence() (in module sndlib)}

\begin{fulllineitems}
\phantomsection\label{index:sndlib.makeSilence}\pysiglinewithargsret{\code{sndlib.}\bfcode{makeSilence}}{\emph{duration}, \emph{fs}}{}
Generate a silence.

This function just fills an array with zeros for the
desired duration.
\begin{description}
\item[{duration}] \leavevmode{[}float{]}
Duration of the silence in milliseconds.

\item[{fs}] \leavevmode{[}int{]}
Samplig frequency in Hz.

\end{description}
\begin{description}
\item[{snd}] \leavevmode{[}2-dimensional array of floats{]}
The array has dimensions (nSamples, 2).

\end{description}

\begin{Verbatim}[commandchars=\\\{\}]
\PYG{g+gp}{\PYGZgt{}\PYGZgt{}\PYGZgt{} }\PYG{n}{sil} \PYG{o}{=} \PYG{n}{makeSilence}\PYG{p}{(}\PYG{n}{duration}\PYG{o}{=}\PYG{l+m+mi}{200}\PYG{p}{,} \PYG{n}{fs}\PYG{o}{=}\PYG{l+m+mi}{48000}\PYG{p}{)}
\end{Verbatim}

\end{fulllineitems}

\index{makeSimpleDichotic() (in module sndlib)}

\begin{fulllineitems}
\phantomsection\label{index:sndlib.makeSimpleDichotic}\pysiglinewithargsret{\code{sndlib.}\bfcode{makeSimpleDichotic}}{\emph{F0}, \emph{lowHarm}, \emph{highHarm}, \emph{cmpLevel}, \emph{lowFreq}, \emph{highFreq}, \emph{spacing}, \emph{sigBandwidth}, \emph{phaseRelationship}, \emph{dichoticDifference}, \emph{itd}, \emph{ipd}, \emph{narrowBandCmpLevel}, \emph{duration}, \emph{ramp}, \emph{fs}, \emph{maxLevel}}{}
Generate harmonically related dichotic pitches, or equivalent
harmonically related narrowband tones in noise.

This function generates first a pink noise by adding closely spaced
sinusoids in a wide frequency range. Then, it can apply an interaural
time difference (ITD), an interaural phase difference (IPD) or a
level increase to harmonically related narrow frequency bands
within the noise. In the first two cases (ITD and IPD) the result
is a dichotic pitch. In the last case the pitch can also be heard
monaurally; adjusting the level increase its salience can be closely
matched to that of a dichotic pitch.
\begin{description}
\item[{F0}] \leavevmode{[}float{]}
Centre frequency of the fundamental in hertz.

\item[{lowHarm}] \leavevmode{[}int{]}
Lowest harmonic component number.

\item[{highHarm}] \leavevmode{[}int{]}
Highest harmonic component number.

\item[{cmpLevel}] \leavevmode{[}float{]}
Level of each sinusoidal frequency component of the noise.

\item[{lowFreq}] \leavevmode{[}float{]}
Lowest frequency in hertz of the noise.

\item[{highFreq}] \leavevmode{[}float{]}
Highest frequency in hertz of the noise.

\item[{spacing}] \leavevmode{[}float{]}
Spacing in cents between the sinusoidal components used to generate the
noise.

\item[{sigBandwidth}] \leavevmode{[}float{]}
Width in cents of each harmonically related frequency band.

\item[{phaseRelationship}] \leavevmode{[}string (`NoSpi' or `NpiSo'){]}
If NoSpi, the phase of the regions within each frequency band will
be shifted. If NpiSo, the phase of the regions between each
frequency band will be shifted.

\item[{dichoticDifference}] \leavevmode{[}string (one of `IPD', `ITD', `Level'){]}
The manipulation to apply to the heramonically related
frequency bands.

\item[{itd}] \leavevmode{[}float{]}
Interaural time difference in microseconds to apply
to the harmonically related frequency bands. Applied
only if `dichoticDifference' is `ITD'.

\item[{ipd}] \leavevmode{[}float{]}
Interaural phase difference in radians to apply
to the harmonically related frequency bands. Applied
only if `dichoticDifference' is `IPD'.

\item[{narrowBandCmpLevel}] \leavevmode{[}float{]}
Level of the sinusoidal components in the frequency bands.
If the `narrowBandCmpLevel' is greater than the level
of the background noise (`cmpLevel'), a complex tone
consisting of narrowband noises in noise will be generated.

\item[{duration}] \leavevmode{[}float{]}
Sound duration (excluding ramps) in milliseconds.

\item[{ramp}] \leavevmode{[}float{]}
Duration of the onset and offset ramps in milliseconds.
The total duration of the sound will be duration+ramp*2.

\item[{fs}] \leavevmode{[}int{]}
Samplig frequency in Hz.

\item[{maxLevel}] \leavevmode{[}float{]}
Level in dB SPL output by the soundcard for a sinusoid of amplitude 1.

\end{description}
\begin{description}
\item[{snd}] \leavevmode{[}2-dimensional array of floats{]}
The array has dimensions (nSamples, 2).

\end{description}

\begin{Verbatim}[commandchars=\\\{\}]
\PYG{g+gp}{\PYGZgt{}\PYGZgt{}\PYGZgt{} }\PYG{n}{s1} \PYG{o}{=} \PYG{n}{makeSimpleDichotic}\PYG{p}{(}\PYG{n}{F0}\PYG{o}{=}\PYG{l+m+mi}{250}\PYG{p}{,} \PYG{n}{lowHarm}\PYG{o}{=}\PYG{l+m+mi}{1}\PYG{p}{,} \PYG{n}{highHarm}\PYG{o}{=}\PYG{l+m+mi}{3}\PYG{p}{,} \PYG{n}{cmpLevel}\PYG{o}{=}\PYG{l+m+mi}{30}\PYG{p}{,}
\PYG{g+go}{    lowFreq=40, highFreq=1200, spacing=10, sigBandwidth=100,}
\PYG{g+go}{    phaseRelationship='NoSpi', dichoticDifference='IPD', itd=0,}
\PYG{g+go}{    ipd=3.14, narrowBandCmpLevel=0, duration=280, ramp=10,}
\PYG{g+go}{    fs=48000, maxLevel=100)}
\end{Verbatim}

Keyword arguments:
F0 -- Fundamental frequency (Hz)
lowHarm -- Number of the lowest harmonic
highHarm -- Number of the highest harmonic
cmpLevel -- level in dB SPL of each sinusoid that makes up the noise
lowCmp -- lowest frequency (Hz)
highCmp -- highest frequency (Hz)
spacing -- spacing between frequency components (Cents)
sigBandwidth -- bandwidth of each harmonic band (Cents)
phaseRelationship -- NoSpi or NpiSo
dichotic difference -- IPD, ITD or Level
itd -- interaural time difference microseconds
ipd -- interaural phase difference in radians
narrowBandCmpLevel - level of frequency components in the harmonic bands (valid only if dichotic difference is Level)
duration -- duration (excluding ramps) in ms
ramp -- ramp duration in ms
fs -- sampling frequency
maxLevel --

\end{fulllineitems}

\index{nextpow2() (in module sndlib)}

\begin{fulllineitems}
\phantomsection\label{index:sndlib.nextpow2}\pysiglinewithargsret{\code{sndlib.}\bfcode{nextpow2}}{\emph{x}}{}
Next power of two.
\begin{description}
\item[{x}] \leavevmode{[}int{]}
Base number.

\end{description}
\begin{description}
\item[{out}] \leavevmode{[}float{]}
The power to which 2 should be raised.

\end{description}

\begin{Verbatim}[commandchars=\\\{\}]
\PYG{g+gp}{\PYGZgt{}\PYGZgt{}\PYGZgt{} }\PYG{n}{nextpow2}\PYG{p}{(}\PYG{l+m+mi}{511}\PYG{p}{)}
\PYG{g+go}{9}
\PYG{g+gp}{\PYGZgt{}\PYGZgt{}\PYGZgt{} }\PYG{l+m+mi}{2}\PYG{o}{*}\PYG{o}{*}\PYG{l+m+mi}{9}
\PYG{g+go}{512}
\end{Verbatim}

\end{fulllineitems}

\index{phaseShift() (in module sndlib)}

\begin{fulllineitems}
\phantomsection\label{index:sndlib.phaseShift}\pysiglinewithargsret{\code{sndlib.}\bfcode{phaseShift}}{\emph{sig}, \emph{f1}, \emph{f2}, \emph{phase\_shift}, \emph{channel}, \emph{fs}}{}
Shift the phases of a sound within a given frequency region.
\begin{description}
\item[{sig}] \leavevmode{[}array of floats{]}
Input signal.

\item[{f1}] \leavevmode{[}float{]}
The start point of the frequency region to be
phase-shifted in hertz.

\item[{f2}] \leavevmode{[}float{]}
The end point of the frequency region to be
phase-shifted in hertz.

\item[{phase\_shift}] \leavevmode{[}float{]}
The amount of phase shift in radians.

\item[{channel}] \leavevmode{[}string (one of `Right', `Left' or `Both'){]}
The channel in which to apply the phase shift.

\item[{fs}] \leavevmode{[}float{]}
The sampling frequency of the sound.

\end{description}

out : 2-dimensional array of floats

\begin{Verbatim}[commandchars=\\\{\}]
\PYG{g+gp}{\PYGZgt{}\PYGZgt{}\PYGZgt{} }\PYG{n}{noise} \PYG{o}{=} \PYG{n}{broadbandNoise}\PYG{p}{(}\PYG{n}{spectrumLevel}\PYG{o}{=}\PYG{l+m+mi}{40}\PYG{p}{,} \PYG{n}{duration}\PYG{o}{=}\PYG{l+m+mi}{180}\PYG{p}{,} \PYG{n}{ramp}\PYG{o}{=}\PYG{l+m+mi}{10}\PYG{p}{,}
\PYG{g+gp}{... }    \PYG{n}{channel}\PYG{o}{=}\PYG{l+s}{'}\PYG{l+s}{Both}\PYG{l+s}{'}\PYG{p}{,} \PYG{n}{fs}\PYG{o}{=}\PYG{l+m+mi}{48000}\PYG{p}{,} \PYG{n}{maxLevel}\PYG{o}{=}\PYG{l+m+mi}{100}\PYG{p}{)}
\PYG{g+gp}{\PYGZgt{}\PYGZgt{}\PYGZgt{} }\PYG{n}{hp} \PYG{o}{=} \PYG{n}{phaseShift}\PYG{p}{(}\PYG{n}{sig}\PYG{o}{=}\PYG{n}{noise}\PYG{p}{,} \PYG{n}{f1}\PYG{o}{=}\PYG{l+m+mi}{500}\PYG{p}{,} \PYG{n}{f2}\PYG{o}{=}\PYG{l+m+mi}{600}\PYG{p}{,} \PYG{n}{phase\PYGZus{}shift}\PYG{o}{=}\PYG{l+m+mf}{3.14}\PYG{p}{,}
\PYG{g+go}{        channel='Left', fs=48000) \PYGZsh{}this generates a Dichotic Pitch}
\end{Verbatim}

\end{fulllineitems}

\index{pinkNoiseFromSin() (in module sndlib)}

\begin{fulllineitems}
\phantomsection\label{index:sndlib.pinkNoiseFromSin}\pysiglinewithargsret{\code{sndlib.}\bfcode{pinkNoiseFromSin}}{\emph{compLevel}, \emph{lowCmp}, \emph{highCmp}, \emph{spacing}, \emph{duration}, \emph{ramp}, \emph{channel}, \emph{fs}, \emph{maxLevel}}{}
Generate a pink noise by adding sinusoids spaced by a fixed
interval in cents.
\begin{description}
\item[{compLevel}] \leavevmode{[}float{]}
Level of each sinusoidal component in dB SPL.

\item[{lowCmp}] \leavevmode{[}float{]}
Frequency of the lowest noise component in hertz.

\item[{highCmp}] \leavevmode{[}float{]}
Frequency of the highest noise component in hertz.

\item[{spacing}] \leavevmode{[}float{]}
Spacing between the frequencies of the sinusoidal components
in hertz.

\item[{duration}] \leavevmode{[}float{]}
Noise duration (excluding ramps) in milliseconds.

\item[{ramp}] \leavevmode{[}float{]}
Duration of the onset and offset ramps in milliseconds.
The total duration of the sound will be duration+ramp*2.

\item[{channel}] \leavevmode{[}string (`Right', `Left' or `Both'){]}
Channel in which the noise will be generated.

\item[{fs}] \leavevmode{[}int{]}
Samplig frequency in Hz.

\item[{maxLevel}] \leavevmode{[}float{]}
Level in dB SPL output by the soundcard for a sinusoid of amplitude 1.

\end{description}
\begin{description}
\item[{snd}] \leavevmode{[}2-dimensional array of floats{]}
The array has dimensions (nSamples, 2).

\end{description}

\begin{Verbatim}[commandchars=\\\{\}]
\PYG{g+gp}{\PYGZgt{}\PYGZgt{}\PYGZgt{} }\PYG{n}{noise} \PYG{o}{=} \PYG{n}{pinkNoiseFromSin}\PYG{p}{(}\PYG{n}{compLevel}\PYG{o}{=}\PYG{l+m+mi}{23}\PYG{p}{,} \PYG{n}{lowCmp}\PYG{o}{=}\PYG{l+m+mi}{100}\PYG{p}{,} \PYG{n}{highCmp}\PYG{o}{=}\PYG{l+m+mi}{1000}\PYG{p}{,}
\PYG{g+go}{    spacing=20, duration=180, ramp=10, channel='Both',}
\PYG{g+go}{    fs=48000, maxLevel=100)}
\end{Verbatim}

\end{fulllineitems}

\index{pinkNoiseFromSin2() (in module sndlib)}

\begin{fulllineitems}
\phantomsection\label{index:sndlib.pinkNoiseFromSin2}\pysiglinewithargsret{\code{sndlib.}\bfcode{pinkNoiseFromSin2}}{\emph{compLevel}, \emph{lowCmp}, \emph{highCmp}, \emph{spacing}, \emph{duration}, \emph{ramp}, \emph{channel}, \emph{fs}, \emph{maxLevel}}{}
Generate a pink noise by adding sinusoids spaced by a fixed
interval in cents.

This function should produce the same output of pinkNoiseFromSin,
it simply uses a different algorithm that uses matrix operations
instead of a for loop. It doesn't seem to be much faster though.
\begin{description}
\item[{compLevel}] \leavevmode{[}float{]}
Level of each sinusoidal component in dB SPL.

\item[{lowCmp}] \leavevmode{[}float{]}
Frequency of the lowest noise component in hertz.

\item[{highCmp}] \leavevmode{[}float{]}
Frequency of the highest noise component in hertz.

\item[{spacing}] \leavevmode{[}float{]}
Spacing between the frequencies of the sinusoidal components
in hertz.

\item[{duration}] \leavevmode{[}float{]}
Noise duration (excluding ramps) in milliseconds.

\item[{ramp}] \leavevmode{[}float{]}
Duration of the onset and offset ramps in milliseconds.
The total duration of the sound will be duration+ramp*2.

\item[{channel}] \leavevmode{[}string (`Right', `Left' or `Both'){]}
Channel in which the noise will be generated.

\item[{fs}] \leavevmode{[}int{]}
Samplig frequency in Hz.

\item[{maxLevel}] \leavevmode{[}float{]}
Level in dB SPL output by the soundcard for a sinusoid of amplitude 1.

\end{description}
\begin{description}
\item[{snd}] \leavevmode{[}2-dimensional array of floats{]}
The array has dimensions (nSamples, 2).

\end{description}

\begin{Verbatim}[commandchars=\\\{\}]
\PYG{g+gp}{\PYGZgt{}\PYGZgt{}\PYGZgt{} }\PYG{n}{noise} \PYG{o}{=} \PYG{n}{pinkNoiseFromSin2}\PYG{p}{(}\PYG{n}{compLevel}\PYG{o}{=}\PYG{l+m+mi}{23}\PYG{p}{,} \PYG{n}{lowCmp}\PYG{o}{=}\PYG{l+m+mi}{100}\PYG{p}{,} \PYG{n}{highCmp}\PYG{o}{=}\PYG{l+m+mi}{1000}\PYG{p}{,}
\PYG{g+go}{    spacing=20, duration=180, ramp=10, channel='Both',}
\PYG{g+go}{    fs=48000, maxLevel=100)}
\end{Verbatim}

\end{fulllineitems}

\index{pureTone() (in module sndlib)}

\begin{fulllineitems}
\phantomsection\label{index:sndlib.pureTone}\pysiglinewithargsret{\code{sndlib.}\bfcode{pureTone}}{\emph{frequency}, \emph{phase}, \emph{level}, \emph{duration}, \emph{ramp}, \emph{channel}, \emph{fs}, \emph{maxLevel}}{}
Synthetise a pure tone.
\begin{description}
\item[{frequency}] \leavevmode{[}float{]}
Tone frequency in hertz.

\item[{phase}] \leavevmode{[}float{]}
Starting phase in radians.

\item[{level}] \leavevmode{[}float{]}
Tone level in dB SPL.

\item[{duration}] \leavevmode{[}float{]}
Tone duration (excluding ramps) in milliseconds.

\item[{ramp}] \leavevmode{[}float{]}
Duration of the onset and offset ramps in milliseconds.
The total duration of the sound will be duration+ramp*2.

\item[{channel}] \leavevmode{[}string (`Right', `Left' or `Both'){]}
Channel in which the tone will be generated.

\item[{fs}] \leavevmode{[}int{]}
Samplig frequency in Hz.

\item[{maxLevel}] \leavevmode{[}float{]}
Level in dB SPL output by the soundcard for a sinusoid of amplitude 1.

\end{description}
\begin{description}
\item[{snd}] \leavevmode{[}2-dimensional array of floats{]}
The array has dimensions (nSamples, 2).

\end{description}

\begin{Verbatim}[commandchars=\\\{\}]
\PYG{g+gp}{\PYGZgt{}\PYGZgt{}\PYGZgt{} }\PYG{n}{pt} \PYG{o}{=} \PYG{n}{pureTone}\PYG{p}{(}\PYG{n}{frequency}\PYG{o}{=}\PYG{l+m+mi}{440}\PYG{p}{,} \PYG{n}{phase}\PYG{o}{=}\PYG{l+m+mi}{0}\PYG{p}{,} \PYG{n}{level}\PYG{o}{=}\PYG{l+m+mi}{65}\PYG{p}{,} \PYG{n}{duration}\PYG{o}{=}\PYG{l+m+mi}{180}\PYG{p}{,}
\PYG{g+gp}{... }    \PYG{n}{ramp}\PYG{o}{=}\PYG{l+m+mi}{10}\PYG{p}{,} \PYG{n}{channel}\PYG{o}{=}\PYG{l+s}{'}\PYG{l+s}{Right}\PYG{l+s}{'}\PYG{p}{,} \PYG{n}{fs}\PYG{o}{=}\PYG{l+m+mi}{48000}\PYG{p}{,} \PYG{n}{maxLevel}\PYG{o}{=}\PYG{l+m+mi}{100}\PYG{p}{)}
\PYG{g+gp}{\PYGZgt{}\PYGZgt{}\PYGZgt{} }\PYG{n}{pt}\PYG{o}{.}\PYG{n}{shape}
\PYG{g+go}{(9600, 2)}
\end{Verbatim}

\end{fulllineitems}

\index{scale() (in module sndlib)}

\begin{fulllineitems}
\phantomsection\label{index:sndlib.scale}\pysiglinewithargsret{\code{sndlib.}\bfcode{scale}}{\emph{level}, \emph{sig}}{}
Increase or decrease the amplitude of a sound signal.
\begin{description}
\item[{level}] \leavevmode{[}float{]}
Desired increment or decrement in dB SPL.

\item[{signal}] \leavevmode{[}array of floats{]}
Signal to scale.

\end{description}

sig : 2-dimensional array of floats

\begin{Verbatim}[commandchars=\\\{\}]
\PYG{g+gp}{\PYGZgt{}\PYGZgt{}\PYGZgt{} }\PYG{n}{noise} \PYG{o}{=} \PYG{n}{broadbandNoise}\PYG{p}{(}\PYG{n}{spectrumLevel}\PYG{o}{=}\PYG{l+m+mi}{40}\PYG{p}{,} \PYG{n}{duration}\PYG{o}{=}\PYG{l+m+mi}{180}\PYG{p}{,} \PYG{n}{ramp}\PYG{o}{=}\PYG{l+m+mi}{10}\PYG{p}{,}
\PYG{g+gp}{... }    \PYG{n}{channel}\PYG{o}{=}\PYG{l+s}{'}\PYG{l+s}{Both}\PYG{l+s}{'}\PYG{p}{,} \PYG{n}{fs}\PYG{o}{=}\PYG{l+m+mi}{48000}\PYG{p}{,} \PYG{n}{maxLevel}\PYG{o}{=}\PYG{l+m+mi}{100}\PYG{p}{)}
\PYG{g+gp}{\PYGZgt{}\PYGZgt{}\PYGZgt{} }\PYG{n}{noise} \PYG{o}{=} \PYG{n}{scale}\PYG{p}{(}\PYG{n}{level}\PYG{o}{=}\PYG{o}{-}\PYG{l+m+mi}{10}\PYG{p}{,} \PYG{n}{sig}\PYG{o}{=}\PYG{n}{noise}\PYG{p}{)} \PYG{c}{\PYGZsh{}reduce level by 10 dB}
\end{Verbatim}

\end{fulllineitems}

\index{steepNoise() (in module sndlib)}

\begin{fulllineitems}
\phantomsection\label{index:sndlib.steepNoise}\pysiglinewithargsret{\code{sndlib.}\bfcode{steepNoise}}{\emph{frequency1}, \emph{frequency2}, \emph{level}, \emph{duration}, \emph{ramp}, \emph{channel}, \emph{fs}, \emph{maxLevel}}{}
Synthetise band-limited noise from the addition of random-phase
sinusoids.
\begin{description}
\item[{frequency1}] \leavevmode{[}float{]}
Start frequency of the noise.

\item[{frequency2}] \leavevmode{[}float{]}
End frequency of the noise.

\item[{level}] \leavevmode{[}float{]}
Noise spectrum level.

\item[{duration}] \leavevmode{[}float{]}
Tone duration (excluding ramps) in milliseconds.

\item[{ramp}] \leavevmode{[}float{]}
Duration of the onset and offset ramps in milliseconds.
The total duration of the sound will be duration+ramp*2.

\item[{channel}] \leavevmode{[}string (`Right', `Left' or `Both'){]}
Channel in which the tone will be generated.

\item[{fs}] \leavevmode{[}int{]}
Samplig frequency in Hz.

\item[{maxLevel}] \leavevmode{[}float{]}
Level in dB SPL output by the soundcard for a sinusoid of amplitude 1.

\end{description}
\begin{description}
\item[{snd}] \leavevmode{[}2-dimensional array of floats{]}
The array has dimensions (nSamples, 2).

\end{description}

\begin{Verbatim}[commandchars=\\\{\}]
\PYG{g+gp}{\PYGZgt{}\PYGZgt{}\PYGZgt{} }\PYG{n}{nbNoise} \PYG{o}{=} \PYG{n}{steepNoise}\PYG{p}{(}\PYG{n}{frequency}\PYG{o}{=}\PYG{l+m+mi}{440}\PYG{p}{,} \PYG{n}{frequency2}\PYG{o}{=}\PYG{l+m+mi}{660}\PYG{p}{,} \PYG{n}{level}\PYG{o}{=}\PYG{l+m+mi}{65}\PYG{p}{,}
\PYG{g+gp}{... }    \PYG{n}{duration}\PYG{o}{=}\PYG{l+m+mi}{180}\PYG{p}{,} \PYG{n}{ramp}\PYG{o}{=}\PYG{l+m+mi}{10}\PYG{p}{,} \PYG{n}{channel}\PYG{o}{=}\PYG{l+s}{'}\PYG{l+s}{Right}\PYG{l+s}{'}\PYG{p}{,} \PYG{n}{fs}\PYG{o}{=}\PYG{l+m+mi}{48000}\PYG{p}{,} \PYG{n}{maxLevel}\PYG{o}{=}\PYG{l+m+mi}{100}\PYG{p}{)}
\end{Verbatim}

\end{fulllineitems}



\chapter{\texttt{pysdt} -- Signal Detection Theory Measures}
\label{index:pysdt-signal-detection-theory-measures}\phantomsection\label{index:module-pysdt}\index{pysdt (module)}
A module for computing signal detection theory measures.
Some of the functions in this module have been ported to
python from the `psyphy' R package of Kenneth Knoblauch
\href{http://cran.r-project.org/web/packages/psyphy/index.html}{http://cran.r-project.org/web/packages/psyphy/index.html}
\index{dprime\_SD() (in module pysdt)}

\begin{fulllineitems}
\phantomsection\label{index:pysdt.dprime_SD}\pysiglinewithargsret{\code{pysdt.}\bfcode{dprime\_SD}}{\emph{H}, \emph{FA}, \emph{meth}}{}
Compute d' for one interval same/different task from `hit' and `false alarm' rates.
\begin{description}
\item[{H}] \leavevmode{[}float{]}
Hit rate.

\item[{FA}] \leavevmode{[}float{]}
False alarms rate.

\item[{meth}] \leavevmode{[}string{]}
`diff' for differencing strategy or `IO' for independent observations strategy.

\end{description}
\begin{description}
\item[{dprime}] \leavevmode{[}float{]}
d' value

\end{description}

\begin{Verbatim}[commandchars=\\\{\}]
\PYG{g+gp}{\PYGZgt{}\PYGZgt{}\PYGZgt{} }\PYG{n}{dp} \PYG{o}{=} \PYG{n}{dprime\PYGZus{}SD}\PYG{p}{(}\PYG{l+m+mf}{0.7}\PYG{p}{,} \PYG{l+m+mf}{0.2}\PYG{p}{,} \PYG{l+s}{'}\PYG{l+s}{IO}\PYG{l+s}{'}\PYG{p}{)}
\end{Verbatim}

\end{fulllineitems}

\index{dprime\_SD\_from\_counts() (in module pysdt)}

\begin{fulllineitems}
\phantomsection\label{index:pysdt.dprime_SD_from_counts}\pysiglinewithargsret{\code{pysdt.}\bfcode{dprime\_SD\_from\_counts}}{\emph{nCA}, \emph{nTA}, \emph{nCB}, \emph{nTB}, \emph{meth}, \emph{corr}}{}
Compute d' for one interval same/different task from counts of correct and total responses.
\begin{description}
\item[{nCA}] \leavevmode{[}int{]}
Number of correct responses in `same' trials.

\item[{nTA}] \leavevmode{[}int{]}
Total number of `same' trials.

\item[{nCB}] \leavevmode{[}int{]}
Number of correct responses in `different' trials.

\item[{nTB}] \leavevmode{[}int{]}
Total number of `different' trials.

\item[{meth}] \leavevmode{[}string{]}
`diff' for differencing strategy or `IO' for independent observations strategy.

\item[{corr}] \leavevmode{[}logical{]}
if True, apply the correction to avoid hit and false alarm rates of 0 or one.

\end{description}
\begin{description}
\item[{dprime}] \leavevmode{[}float{]}
d' value

\end{description}

\begin{Verbatim}[commandchars=\\\{\}]
\PYG{g+gp}{\PYGZgt{}\PYGZgt{}\PYGZgt{} }\PYG{n}{dp} \PYG{o}{=} \PYG{n}{dprime\PYGZus{}SD}\PYG{p}{(}\PYG{l+m+mf}{0.7}\PYG{p}{,} \PYG{l+m+mf}{0.2}\PYG{p}{,} \PYG{l+s}{'}\PYG{l+s}{IO}\PYG{l+s}{'}\PYG{p}{)}
\end{Verbatim}

\end{fulllineitems}

\index{dprime\_mAFC() (in module pysdt)}

\begin{fulllineitems}
\phantomsection\label{index:pysdt.dprime_mAFC}\pysiglinewithargsret{\code{pysdt.}\bfcode{dprime\_mAFC}}{\emph{Pc}, \emph{m}}{}
Compute d' corresponding to a certain proportion of correct
responses in m-AFC tasks.
\begin{description}
\item[{Pc}] \leavevmode{[}float{]}
Proportion of correct responses.

\item[{m}] \leavevmode{[}int{]}
Number of alternatives.

\end{description}
\begin{description}
\item[{dprime}] \leavevmode{[}float{]}
d' value

\end{description}

\begin{Verbatim}[commandchars=\\\{\}]
\PYG{g+gp}{\PYGZgt{}\PYGZgt{}\PYGZgt{} }\PYG{n}{dp} \PYG{o}{=} \PYG{n}{dprime\PYGZus{}mAFC}\PYG{p}{(}\PYG{l+m+mf}{0.7}\PYG{p}{,} \PYG{l+m+mi}{3}\PYG{p}{)}
\end{Verbatim}

\end{fulllineitems}

\index{dprime\_yes\_no() (in module pysdt)}

\begin{fulllineitems}
\phantomsection\label{index:pysdt.dprime_yes_no}\pysiglinewithargsret{\code{pysdt.}\bfcode{dprime\_yes\_no}}{\emph{H}, \emph{FA}}{}
Compute d' for one interval `yes/no' type tasks from hits and false alarm rates.
\begin{description}
\item[{H}] \leavevmode{[}float{]}
Hit rate.

\item[{FA}] \leavevmode{[}float{]}
False alarms rate.

\end{description}
\begin{description}
\item[{dprime}] \leavevmode{[}float{]}
d' value

\end{description}

\begin{Verbatim}[commandchars=\\\{\}]
\PYG{g+gp}{\PYGZgt{}\PYGZgt{}\PYGZgt{} }\PYG{n}{dp} \PYG{o}{=} \PYG{n}{dprime\PYGZus{}yes\PYGZus{}no}\PYG{p}{(}\PYG{l+m+mf}{0.7}\PYG{p}{,} \PYG{l+m+mf}{0.2}\PYG{p}{)}
\end{Verbatim}

\end{fulllineitems}

\index{dprime\_yes\_no\_from\_counts() (in module pysdt)}

\begin{fulllineitems}
\phantomsection\label{index:pysdt.dprime_yes_no_from_counts}\pysiglinewithargsret{\code{pysdt.}\bfcode{dprime\_yes\_no\_from\_counts}}{\emph{nCA}, \emph{nTA}, \emph{nCB}, \emph{nTB}, \emph{corr}}{}
Compute d' for one interval `yes/no' type tasks from counts of correct and total responses.
\begin{description}
\item[{nCA}] \leavevmode{[}int{]}
Number of correct responses in `signal' trials.

\item[{nTA}] \leavevmode{[}int{]}
Total number of `signal' trials.

\item[{nCB}] \leavevmode{[}int{]}
Number of correct responses in `noise' trials.

\item[{nTB}] \leavevmode{[}int{]}
Total number of `noise' trials.

\item[{corr}] \leavevmode{[}logical{]}
if True, apply the correction to avoid hit and false alarm rates of 0 or one.

\end{description}
\begin{description}
\item[{dprime}] \leavevmode{[}float{]}
d' value

\end{description}

\begin{Verbatim}[commandchars=\\\{\}]
\PYG{g+gp}{\PYGZgt{}\PYGZgt{}\PYGZgt{} }\PYG{n}{dp} \PYG{o}{=} \PYG{n}{dprime\PYGZus{}yes\PYGZus{}no\PYGZus{}from\PYGZus{}counts}\PYG{p}{(}\PYG{n}{nCA}\PYG{o}{=}\PYG{l+m+mi}{70}\PYG{p}{,} \PYG{n}{nTA}\PYG{o}{=}\PYG{l+m+mi}{100}\PYG{p}{,} \PYG{n}{nCB}\PYG{o}{=}\PYG{l+m+mi}{80}\PYG{p}{,} \PYG{n}{nTB}\PYG{o}{=}\PYG{l+m+mi}{100}\PYG{p}{,} \PYG{n}{corr}\PYG{o}{=}\PYG{n+nb+bp}{True}\PYG{p}{)}
\end{Verbatim}

\end{fulllineitems}



\chapter{Default Experiments}
\label{index:default-experiments}
\emph{audiogram}

\emph{audiogram\_mf}

\emph{freq}


\renewcommand{\indexname}{Python Module Index}
\begin{theindex}
\def\bigletter#1{{\Large\sffamily#1}\nopagebreak\vspace{1mm}}
\bigletter{p}
\item {\texttt{pysdt}}, \pageref{index:module-pysdt}
\indexspace
\bigletter{s}
\item {\texttt{sndlib}}, \pageref{index:module-sndlib}
\end{theindex}

\renewcommand{\indexname}{Index}
\printindex
\end{document}
